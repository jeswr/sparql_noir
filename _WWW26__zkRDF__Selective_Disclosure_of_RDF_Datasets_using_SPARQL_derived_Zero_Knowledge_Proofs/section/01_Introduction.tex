\section{Introduction}
\label{sec:intro}

Sharing data via the Web\todo{JW: Removed "the Web" because VCs are increasingly transported by BLE etc.} -- business, product, personal data -- is increasingly requiring cryptographic assurance of shared information:
Electronic signatures on business reports~\cite{PwC2018digitalisation} or prescriptions in healthcare~\cite{Kierkegaard2013} have been common-place for years, 
the recent regulatory mandate for Digital Product Passports aims to foster transparency~\cite{Bureau2025DataDriven}, and the EUDI Wallet~\cite{euIDwallet} will transform how EU citizens share attested personal information with organizations.
In sharing such attested data, the twin issues of data integrity and data privacy arise:
On one hand, ensuring cryptographic data integrity when sharing data helps to combat fraud, to improve data quality and to create accountability~\cite{NAPA2025DataSharing}.
On the other hand, compliance or regulatory barriers, especially concerning the EU's GDPR~\cite{Graux2024Symbiosis}, mandate ensuring data privacy to protect \eg citizens' private information or business secrets.

Consider Figure~\ref{fig:example}: 
To receive a business loan, a bank requires proof that an applicant, \eg freelance consultant Alice, has sufficiently stable income to cover the monthly payments. 
Instead of plainly submitting sensitive tax filings, which were prepared and digitally signed by Alice's tax advisor, Alice only needs to prove to the bank that (a) the income is above the required threshold and (b) that the income was signed off as correct by the tax advisor.
The questions follow immediately:
How can the bank, the data consumer, specify which attested information they need to receive from Alice, the data holder?
And, how can then the data holder avoid disclosing any information beyond the required while proving the integrity it?

\begin{figure*}[t!]
    \centering
    \includegraphics[width=0.775\linewidth]{img/SSI-ZKP-ESWC-2026.png}
    \caption{The problem: Upon request, prove required data but nothing beyond.
    }
    \Description{A diagram illustrating the process of a data holder (Alice) deriving a presentation with proof from a digitally signed tax filing, in response to a bank's request for proof of income (income > 50K). The tax advisor issues the digitally signed tax filings, and the bank verifies the proof. The dashed lines indicate trust relationships and requests.}
    \label{fig:example}
\end{figure*}


To strike such a balance, Europe's leading cryptographers strongly advocate for the application of Zero-Knowledge Proofs (ZKPs) on top of\todo{JW: alongside} digital signatures~\cite{cryptoFeedback}.
Using a ZKP, a prover is able to convince a verifier that a claim is true without the verifier learning any additional information~\cite{DBLP:conf/stoc/GoldwasserMR85}.
To model asserted data and cryptographic provenance information, the W3C Verifiable Credentials (VCs)~\cite{VC} recommends a graph-based data model based on the Resource Description Framework (RDF)~\cite{RDF}.
Previous efforts on Linked Data Integrity Proofs~\cite{DataIntegrityBBS} were recently complemented by a definition of RDF-based semantics for selective disclosure~\cite{DBLP:conf/esws/BraunK25}.
It is logically sound to apply querying and reasoning techniques on selectively disclosing RDF datasets~\cite{DBLP:conf/esws/BraunK25}, \eg VCs and their presentations, even when applying ZKPs.
This logical consistency~\cite{DBLP:conf/esws/BraunK25} enables the central idea of our approach: using SPARQL~\cite{SPARQL}, the standard query language for RDF, to express which data or properties thereof to disclose.

In this paper, we introduce \textit{zkRDF}, a data-centric approach for selectively disclosing RDF datasets based on SPARQL-derived Zero-Knowledge Proofs.
Upon receiving a SPARQL query from a data consumer, a data holder executes the received query and discloses -- not the results -- but a subset of the queried dataset with attached proofs about the desired properties as expressed by the query.
Such a presentation of the queried dataset including corresponding proofs is called a \enquote{selectively disclosing dataset}~\cite{DBLP:conf/esws/BraunK25}.
So, rather than the data holder creating a proof that the query was executed correctly, the verifier receives a selectively disclosing dataset from which they are able to obtain the query results themselves.
% Both approaches result in the same data disclosure - and are thus equally privacy preserving.
There is no monolithic proof about SPARQL solution mappings; the produced proof is composed of sub-proofs about the underlying RDF dataset. 

Our evaluation shows that most SPARQL features are supported in \textit{zkRDF} while the remaining cannot directly be supported without privacy caveats.
In terms of performance, we compared our proof-of-concept (PoC) implementation\footnote{\url{https://anonymous.4open.science/r/rdf-zkp-sparql}} to a prototype following an alternative but comparable approach.
Their approach relies on a zero-knowledge virtual machine and is shown to be three orders of magnitude slower on their own benchmark than our approach.% We acknowledge that other, more performant, options for proving properties of computations are available - such as using circuit-builders.

%
We highlight our contributions:
\begin{itemize}[noitemsep,topsep=0pt,parsep=0pt,partopsep=0pt,wide,labelwidth=!,labelindent=0pt,leftmargin=14pt]
    \item a first distinction between data-centric and computation-centric\todo{JW: I would instead phrase this as "ZKP of query evaluation" vs "ZKP of data properties"} approaches to proving query results in RDF datasets; based on related work (Section~\ref{sec:related}) 
    \item \textit{zkRDF} (Section~\ref{sec:framework}): abstract methodology and conceptual instantiation for SPARQL-derived selective disclosure of RDF datasets (a formalisation is available in Appendix~\ref{appendix:zkrdf}).
    \item its evaluation (Section~\ref{sec:eval}) along the three dimensions of
    \begin{itemize}[noitemsep,topsep=0pt,parsep=0pt,partopsep=0pt,wide,labelwidth=!,labelindent=0pt,leftmargin=14pt]
        \item competence -- in terms of SPARQL feature support
        \item performance -- to compare implementations to alternatives
        \item privacy and security -- to highlight potential limitations
    \end{itemize}
    providing an initial evaluation framework.
\end{itemize}
To cover the necessary foundations, we first present preliminaries (Section~\ref{sec:prelim}). 
A brief example illustrates the approach from a practical perspective (Section~\ref{sec:example}).
After presentation of the contributions, we conclude the paper with a vision for future research (Section~\ref{sec:conclusion}).







\begin{comment}
On Proof of Completeness (The Prover's Choice)

fundamental feature is holder-centric privacy.
The goal is for the prover to prove they meet a certain requirement, the system not designed to prove that the holder has revealed all possible matching data.
\end{comment}


