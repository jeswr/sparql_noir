\section{A Vision for Zero-Knowledge SPARQL}
\label{sec:conclusion}

We presented zkRDF, a data-centric approach to selectively disclose and prove SPARQL query results in RDF datasets.
% We believe zkRDF's abstract methodology to be general enough to also potentially apply to conceptual instantiations\todo{it  unclear to me what this means} other than the one presented in this paper.
zkRDF introduces the general idea and capability to prove properties about RDF data.

Using SPARQL queries to specify what properties need to be proven comes as a natural extension of RDF-based semantics for selective disclosure~\cite{DBLP:conf/esws/BraunK25}.
The current alternative from the VC domain, the Digital Crednetial Query Langauge, is deliberately designed to restrict what query can be expressed.
On one hand, this protects users from overly eager queries leading to data oversharing; but on the other hand, the low expressivity deniess expressing more complex properties like numeric bound.
In the context of the eIDAS regulation~\cite{eIDAS} for example, qualified trust service providers (qTSP) of electronic attestation of attributes (EAA) might find SPARQL's and thus zkRDF's expressivity useful to provide better services.
And, use cases along the lines of our initial example from Figure~\ref{fig:example}, this expressivity is strictly required to facilitate the use case.

For other more analytical use cases, the data-centric approach of zkRDF may not suffice.
As discussed in Section~\ref{sec:eval_competence}, support of aggregates comes with potential information leakage as pointed out in Section~\ref{sec:eval_security}.
When \eg such information exposure is unacceptable, a computation-centric approach may be the better choice.
Computation-centric approaches to querying RDF datasets in zero-knowledge thus pose as clear future research.
We thus declare our vision: proving SPARQL query results in \textit{true} zero-knowledge.

With our work on zkRDF, 
we add to the emergent research at the intersection of semantic web technologies and privacy-enhancing technology of ZKPs.
Given that sharing data via the Web increasingly requires attestation and assurance, we envision that the trinity of RDF as a data model, SPARQL as the query language and ZKPs as a means of privacy-preserving provenance  will be a valuable part of the solution to foster transparency and privacy, accountability, and -- ultimately -- trust on the Web.





