\section{Illustrating Example}
\label{sec:example}

In this section, we illustrate the data-centric nature of zkRDF to aid presenting the details of the approach in Section~\ref{sec:framework}.

Consider again our initial example from Figure~\ref{fig:example}.
Freelance consultant Alice is issued their tax filings by their tax consultant in form of a digitally signed dataset, \eg, Verifiable Credentials containing the tax filing information modeled in Listing~\ref{listing:run-example-A} for 2024 and Listing~\ref{listing:run-example-B} for 2023.
\begin{table}[!b]
\centering
\begin{tabular}{p{0.46\linewidth} p{0.46\linewidth}}

% Left Column: Listing for Fiscal Year 2024
\lstset{style=turtle}
\begin{lstlisting}[
    language=Turtle, 
    caption={Simplified tax filing (2024); Turtle syntax.},
    label={listing:run-example-A},
    basicstyle=\scriptsize\ttfamily
]
[] a fin:TaxReturn ;
 fin:about ex:alice ;
 fin:provider ex:trustedTaxAdvisor;
 fin:taxPeriod "2024" ;
 fin:taxableIncome [
  a fin:MonetaryAmount ;
  fin:currency "USD" ;
  fin:value "123000"^^xsd:integer
 ] . # signature details omitted
\end{lstlisting}

& % Column separator

% Right Column: Listing for Fiscal Year 2023
\lstset{style=turtle}
\begin{lstlisting}[
    language=Turtle, 
    caption={Simplified tax filing (2023); Turtle syntax.},
    label={listing:run-example-B},
    basicstyle=\scriptsize\ttfamily
]
[] a fin:TaxReturn ;
 fin:about ex:alice ;
 fin:provider ex:trustedTaxAdvisor;
 fin:taxPeriod "2023" ;
 fin:taxableIncome [
  a fin:MonetaryAmount ;
  fin:currency "USD" ;
 fin:value "90000"^^xsd:integer
 ] . # signature details omitted
\end{lstlisting}
\\
\end{tabular}
\end{table}

When Alice asks a bank for a loan, the bank requires certain information to prepare a suitable loan offer.
For example, the bank requires that last year's income is above 85000 (USD).
To receive an offer with a more beneficial interest rate, the bank optionally requests proof that the income in the year before was also above the threshold.
For a first loan offer, the bank does not require a user's identifier -- only financial information are relevant in this stage of the process.
We express such a query using SPARQL (Listing~\ref{listing:run-example-Q}).

\begin{table}[!t]
\centering
\begin{tabular}{p{0.96\linewidth} }
\lstset{style=sparql}
\begin{lstlisting}[
    language=SPARQL, 
    caption={The bank's SPARQL query for an income greater than 85000 (USD) in 2024, and optionally in 2023.} ,
    label={listing:run-example-Q}, 
    basicstyle=\scriptsize\ttfamily
]
SELECT ?taxAdvisor # disclose which tax advisor signed the filing
WHERE {
    BIND(85000 as ?requiredIncome)
    ?return2024     a     fin:TaxReturn ;
        fin:about         ?user ;  # initially, hide user (not projected)
        fin:provider      ?taxAdvisor ; # find tax advisor 
        fin:taxPeriod     "2024" ;
        fin:taxableIncome [ fin:currency "USD" ;
                            fin:value ?income2024 ] .
    FILTER(?income2024 > ?requiredIncome) # check income threshold
    # optionally, to get a better rate provide more information
    OPTIONAL {
    ?return2023     a     fin:TaxReturn ;
        fin:about         ?user ; # require filing for same user
        fin:provider      ?taxAdvisor ; # find tax advisor
        fin:taxPeriod     "2023" ;
        fin:taxableIncome [ fin:currency "USD" ;
                            fin:value ?income2023 ] .
    FILTER(?income2023 > ?requiredIncome)  # check income threshold
}   }

\end{lstlisting}
\\
\end{tabular}
% \end{table}

% \begin{table}[!t]
% \centering
\begin{tabular}{p{0.96\linewidth} }
\lstset{style=turtle}
\begin{lstlisting}[
    language=Turtle, 
    caption={An excerpt of the selectively disclosing RDF dataset that is presented to the bank, \ie the data consumer and verifier. Cryptographic details are omitted for brevity.} ,
    label={listing:run-example-P}, 
    basicstyle=\scriptsize\ttfamily
]
GRAPH _:0_4 {
	_:0_0 a fin:TaxReturn .
	_:0_0 fin:about _:0_7 .
	_:0_0 fin:provider ex:trustedTaxAdvisor .
	_:0_0 fin:taxPeriod "2024" .
	_:0_0 fin:taxableIncome _:0_22 .
    _:0_25 _:0_26 _:0_27 . 
    _:0_22 fin:currency "USD".
    _:0_22 fin:value _:0_37 .
}
GRAPH _:2_4 {
	_:2_0 a fin:TaxReturn .
	_:2_0 fin:about _:0_7 .
	_:2_0 fin:provider ex:trustedTaxAdvisor .
	_:2_0 fin:taxPeriod "2023" .
	_:2_0 fin:taxableIncome _:2_22 .
    _:2_25 _:2_26 _:2_27 . 
    _:2_22 fin:currency "USD".
    _:2_22 fin:value _:2_37 .
}
GRAPH _:presentationProofGraph {
	_:cproof rdf:type zkp:CompositeProof .
	_:cproof zkp:hasComponent _:p0 , _:p1 , _:p2 , _:p3 .
    _:0_7 spok:hasSchnorrResponse "a..f"^^xsd:base64Binary.# ex:alice proven
	_:p0 rdf:type bbsp16:PoKS16 . # proof of knowledge of signature
    _:p0 bbsp16:isProofOfKnowledgeOfSignatureOverGraph _:0_4 .
	_:p1 rdf:type lg16:LegoGroth16ProofOfRangeMembership . # numeric bounds
    _:p1 lg16:hasWitness _:0_37 .
	_:p1 lg16:hasLowerBound "85000"^^xsd:nonNegativeInteger .
	_:p2 rdf:type bbsp16:PoKS16 . # proof of knowledge of signature
	_:p2 bbsp16:isProofOfKnowledgeOfSignatureOverGraph _:2_4 .
    _:p3 rdf:type lg16:LegoGroth16ProofOfRangeMembership . # numeric bounds
    _:p3 lg16:hasWitness _:2_37 .
	_:p3 lg16:hasLowerBound "85000"^^xsd:nonNegativeInteger .
	# more proof details omitted for brevity
}
\end{lstlisting}
\\
\end{tabular}
\end{table}

Upon receiving this query, Alice -- or rather their data management system, \eg their EUDI wallet~\cite{euIDwallet} or Solid Pod~\cite{SolidProtocol} -- searches the available RDF dataset for resulting query results.
A solution is found where the optional information on the 2023 tax filings are included.
To get a better interest rate on their loan, Alice decides to keep these optional bindings in the results.

A selectively disclosing dataset is thus created.
Listing~\ref{listing:run-example-P} shows the RDF dataset to be presented to the bank:
It is comprised of RDF graphs where some information remains hidden -- because it is not required to be disclosed.
Blank nodes serve as stand-ins, and corresponding proofs of knowledge of the underlying original values are attached.
While Alice's identifier remains hidden (see blank node \texttt{\_:0\_7}), the identifier is still proven to occur in both tax filings.
In addition, proofs of knowledge of signatures that show that the revealed and hidden information, \ie the tax filing, had been signed by a particular and known tax advisor.
Finally, proofs of numeric bounds (cf. L27-30, L33-35) show that the attested income from 2024 and from 2023 are indeed above the required threshold.

For the cryptographic foundations, we refer to~\cite{DBLP:conf/esws/BraunK25}. 
We highlight in the context of Listing~\ref{listing:run-example-P} that the term \texttt{\_:0\_7} is re-used across graphs \texttt{\_:0\_4} and \texttt{\_:2\_4}. 
The proof includes the fact that all occurrences of \texttt{\_:0\_7} refers to the same underlying secret value, Alice's identifier. 
Similarly, for the proofs of numeric bounds of \texttt{\_:0\_37} and \texttt{\_:2\_37} indicate that the particular values that were indeed signed by the tax advisor -- proven by the proof of knowledge of signatures over graphs \texttt{\_:0\_4} and \texttt{\_:2\_4} respectively.
Note that the tax advisor's identifier is revealed, as requested in the \texttt{SELECT} statement of the query.

All other information are only proven to be true:
\begin{itemize}[noitemsep,topsep=0pt,parsep=0pt,partopsep=0pt,wide,labelwidth=!,labelindent=0pt,leftmargin=14pt]
\item[(a)] The signatures on the graphs are proven to be valid, the graphs' triples are proven to be known, without needing to reveal terms.
\item[(b)] The relations between Alice and their tax filing are proven to exist without revealing their identifier.
\item[(c)] The numeric bounds as per \texttt{FILTER} statements are proven to be fulfilled without revealing the literals.
\end{itemize}

Only information explicitly requested (or previously known as the URIs) in the query are revealed, while still being proven to have been signed. For example in graph \texttt{\_:0\_4}, we do not reveal that \texttt{\_:0\_25} actually the same value that also underlies \texttt{\_:0\_22}.
