        % This is samplepaper.tex, a sample chapter demonstrating the
% LLNCS macro package for Springer Computer Science proceedings;
% Version 2.20 of 2017/10/04
%
%\documentclass[runningheads]{llncs}
%\documentclass[sigconf]{acmart} % anonymous,review % screen (for blue hyperlinks)
\documentclass[sigconf]{acmart} 
%\documentclass[sigconf,anonymous,review ]{acmart} 
%% NOTE that a single column version may required for 
%% submission and peer review. This can be done by changing
%% the \doucmentclass[...]{acmart} in this template to 
%% \documentclass[manuscript,screen]{acmart}
%% 
%% To ensure 100% compatibility, please check the white list of
%% approved LaTeX packages to be used with the Master Article Template at
%% https://www.acm.org/publications/taps/whitelist-of-latex-packages 
%% before creating your document. The white list page provides 
%% information on how to submit additional LaTeX packages for 
%% review and adoption.
%% Fonts used in the template cannot be substituted; margin 
%% adjustments are not allowed.

%%
%% \BibTeX command to typeset BibTeX logo in the docs
\AtBeginDocument{%
  \providecommand\BibTeX{{%
  %  \normalfont B\kern-0.5em{\scshape i\kern-0.25em b}\kern-0.8em\TeX}}}
    Bib\TeX}}}
    
%% Rights management information.  This information is sent to you
%% when you complete the rights form.  These commands have SAMPLE
%% values in them; it is your responsibility as an author to replace
%% the commands and values with those provided to you when you
%% complete the rights form.


\copyrightyear{2026}
\acmYear{2026}
\setcopyright{rightsretained}
\acmConference[WWW '26]{Proceedings of the ACM Web Conference 2026}{April 13--17, 2026}{Dubai, UAE}
\acmBooktitle{Proceedings of the ACM Web Conference 2026 (WWW '26), April 13--17, 2026, Dubai, UAE}
\acmDOI{XX.XXX/XXXX.XXXX}
\acmISBN{XXX-X-XXXX-XXXX-X/XX/XX}


% The following includes the CC license icon appropriate for your paper.
% Download the image from www.scomminc.com/pp/acmsig/4ACM-CC-by-nc-sa-88x31.eps
% and place within your figs or figures folder

\makeatletter
\gdef\@copyrightpermission{
  \begin{minipage}{0.3\columnwidth}
   \href{https://creativecommons.org/licenses/by-nc-sa/4.0/} {\includegraphics[width=0.90\textwidth]{img/4ACM-CC-by-nc-sa-88x31.eps}}
  \end{minipage}\hfill
  \begin{minipage}{0.7\columnwidth}
   \href{https://creativecommons.org/licenses/by-nc-sa/4.0/}{This work is licensed under a Creative Commons Attribution-NonCommercial-ShareAlike International 4.0 License.}
  \end{minipage}
  \vspace{5pt}
}
\makeatother


%
\usepackage{graphicx} % ok


\widowpenalty 10000 
\clubpenalty 10000
% are packages ok or not ok for ACM?
\usepackage{xspace} % ok
\usepackage{enumitem} % ok
\usepackage{multirow} % ok
%\usepackage{paralist} % not ok

\usepackage{csquotes}
\usepackage{booktabs}


%\pagenumbering{gobble}
\newcommand{\ie}{i.\,e.\@\xspace} % not sure if ACM allows newcommands this year...
\newcommand{\eg}{e.\,g.\@\xspace} % not sure if ACM allows newcommands this year...
\newcommand{\wrt}{w.\,r.\,t.\@\xspace} % not sure if ACM allows newcommands this year...
\setlength {\marginparwidth }{1.5cm}
\usepackage[colorinlistoftodos,textsize=tiny]{todonotes} 

\newtheorem{assumption}{Assumption}

% For the Pie Charts:
% \usepackage{pgf-pie} % not ok
\usepackage{subcaption} % ok
% \usepackage{changepage} % not ok
%\usepackage{tikz,pgfplots,pgfplotstable} % not ok
%\pgfplotsset{compat=1.9}


% For Turtle Listings:
% Turtle Listings

\usepackage{listings} % ok
\usepackage{xcolor} % ok
\definecolor{codegreen}{rgb}{0,0.6,0}
\definecolor{codegray}{rgb}{0.5,0.5,0.5}
\definecolor{codepurple}{rgb}{0.58,0,0.82}
\definecolor{backcolour}{rgb}{0.95,0.95,0.92}
\lstdefinestyle{turtle}{
    backgroundcolor=\color{backcolour},   
    commentstyle=\color{codegreen},
    keywordstyle=\bfseries,
    numberstyle=\tiny\color{codegray},
    stringstyle=\color{codepurple},
    basicstyle=\ttfamily\footnotesize,
    breakatwhitespace=false,         
    breaklines=true,                 
    captionpos=b,                    
    keepspaces=true,                 
    numbers=left,                    
    numbersep=5pt,                  
    showspaces=false,                
    showstringspaces=false,
    showtabs=false,                  
    tabsize=2
}
\lstdefinelanguage{Turtle}
{
    columns=fullflexible,
    morekeywords={@prefix,@base,@forSome,@forAll,@keywords},
    morecomment=[l]{\#},
    tabsize=4,
    alsoletter={-?}, % allowed in names
    morecomment=[s][\color{blue}]{<}{>},
    %numberstyle=\color{black},
    morestring=[b][\color{codepurple}]\",
}
% --- SPARQL Style Definition ---
\lstdefinestyle{sparql}{
    backgroundcolor=\color{backcolour},   
    commentstyle=\color{codegreen},
    keywordstyle=\bfseries,
    numberstyle=\tiny\color{codegray},
    stringstyle=\color{codepurple},
    basicstyle=\ttfamily\footnotesize,
    breakatwhitespace=false,         
    breaklines=true,                 
    captionpos=b,                    
    keepspaces=true,                 
    numbers=left,                    
    numbersep=5pt,                   
    showspaces=false,                
    showstringspaces=false,
    showtabs=false,                  
    tabsize=2
}
% --- SPARQL Language Definition ---
\lstdefinelanguage{SPARQL}{
    columns=fullflexible,
    morekeywords={
        PREFIX, BASE, SELECT, DISTINCT, CONSTRUCT, DESCRIBE, ASK, WHERE, 
        FROM, NAMED, GRAPH, OPTIONAL, UNION, FILTER, 
        ORDER, BY, ASC, DESC, LIMIT, OFFSET, 
        GROUP, HAVING,
        BIND, VALUES,
        COUNT, SUM, MIN, MAX, AVG, SAMPLE, 
        STR, LANG, DATATYPE, BOUND, REGEX, EXISTS, NOT
    },
    morecomment=[l]{\#},
    morestring=[b]",
    morestring=[b]',
    moredelim=**[s][\color{darkgray}]{?}{\ }, % Variables starting with ?
    moredelim=**[s][\color{darkgray}]{\$}{\ }, % Variables starting with $
    moredelim=[s][\color{blue}]{<}{>},    % URIs
    alsoletter={?} % Allow ? in identifiers for variables
}



\begin{document}
%
% \title{zkRDF: Selective Disclosure of RDF Datasets using SPARQL-derived Zero-Knowledge Proofs}
\title{zkRDF: Proving the Validity of SPARQL Query Results using Selective Disclosure of RDF Datasets and Zero-Knowledge Proofs}
%
%\titlerunning{Abbreviated paper title}
% If the paper title is too long for the running head, you can set
% an abbreviated paper title here
%
\author{Christoph H.-J. Braun}
\email{braun@kit.edu}
\orcid{0000-0002-5843-0316}
\affiliation{%
  \department{Institute AIFB}
  \institution{Karlsruhe Institute of Technology}
  \city{Karlsruhe}
  \country{Germany}
}
\author{Jesse Wright}
\email{jesse.wright@cs.ox.ac.uk}
\orcid{0000-0002-5771-988X}
\affiliation{%
    \department{Department of Computer Science}
  \institution{University of Oxford}
  \city{Oxford}
  \country{United Kingdom}
}
\author{Tobias Käfer}
\email{tobias.kaefer@kit.edu}
\orcid{0000-0003-0576-7457}
\affiliation{%
\department{Institute AIFB}
  \institution{Karlsruhe Institute of Technology}
  \city{Karlsruhe}
  \country{Germany}
}


%
\begin{abstract}
We introduce {zkRDF}, a data-centric approach to selectively disclose SPARQL query results in RDF datasets using Zero-Knowledge Proofs (ZKPs).\todo{JW: soundness of SPARQL query results without disclosing the underlying dataset}
Instead of proving the correct execution of a query, query results are indirectly proven to be correct by proving properties about the underlying RDF dataset: 
The queried RDF dataset is selectively disclosed based on the patterns and expressions specified by the SPARQL query.
A data consumer\todo{JW: I would suggest consistently using issuer/holder/verifier terminology to be consistent with VC ladn}, who verifies the proofs about the properties of the received dataset is then able to obtain the desired query results by executing the query (or a re-written version of it) on the received selectively disclosing dataset.
We present zkRDF's abstract methodology and detail a conceptual instantiation thereof -- how a SPARQL query is to be interpreted to selectively disclose the corresponding query results in the underlying dataset.
We show that zkRDF supports commonly-used SPARQL features and that our proof-of-concept implementation outperforms an alternative approach, which generates ZKPs of SPARQL query computations on their own benchmark significantly.
We also highlight security and privacy considerations for solution architects.
\end{abstract}





%%
%% The code below is generated by the tool at http://dl.acm.org/ccs.cfm.
%% Please copy and paste the code instead of the example below.
%%
\begin{CCSXML}
<ccs2012>
   <concept>
       <concept_id>10002951.10003260</concept_id>
       <concept_desc>Information systems~World Wide Web</concept_desc>
       <concept_significance>500</concept_significance>
       </concept>
 <ccs2012>
<concept>
<concept_id>10002951.10002952.10002953.10010146</concept_id>
<concept_desc>Information systems~Graph-based database models</concept_desc>
<concept_significance>500</concept_significance>
</concept>
<concept>
<concept_id>10002951.10002952.10003197.10010825</concept_id>
<concept_desc>Information systems~Query languages for non-relational engines</concept_desc>
<concept_significance>500</concept_significance>
</concept>
<concept>
<concept_id>10002978.10003029.10011150</concept_id>
<concept_desc>Security and privacy~Privacy protections</concept_desc>
<concept_significance>500</concept_significance>
</concept>
</ccs2012>
\end{CCSXML}

\ccsdesc[500]{Information systems~World Wide Web}
\ccsdesc[500]{Security and privacy~Privacy protections}
\ccsdesc[300]{Information systems~Graph-based database models}
\ccsdesc[300]{Information systems~Query languages for non-relational engines}


%%
%% Keywords. The author(s) should pick words that accurately describe
%% the work being presented. Separate the keywords with commas.
\keywords{RDF, Verifiable Credentials, SPARQL, Zero-Knowledge Proofs}



\maketitle              % typeset the header of the contribution
%
%
%
\section{Introduction}
\label{sec:intro}

Sharing data via the Web\todo{JW: Removed "the Web" because VCs are increasingly transported by BLE etc.} -- business, product, personal data -- is increasingly requiring cryptographic assurance of shared information:
Electronic signatures on business reports~\cite{PwC2018digitalisation} or prescriptions in healthcare~\cite{Kierkegaard2013} have been common-place for years, 
the recent regulatory mandate for Digital Product Passports aims to foster transparency~\cite{Bureau2025DataDriven}, and the EUDI Wallet~\cite{euIDwallet} will transform how EU citizens share attested personal information with organizations.
In sharing such attested data, the twin issues of data integrity and data privacy arise:
On one hand, ensuring cryptographic data integrity when sharing data helps to combat fraud, to improve data quality and to create accountability~\cite{NAPA2025DataSharing}.
On the other hand, compliance or regulatory barriers, especially concerning the EU's GDPR~\cite{Graux2024Symbiosis}, mandate ensuring data privacy to protect \eg citizens' private information or business secrets.

Consider Figure~\ref{fig:example}: 
To receive a business loan, a bank requires proof that an applicant, \eg freelance consultant Alice, has sufficiently stable income to cover the monthly payments. 
Instead of plainly submitting sensitive tax filings, which were prepared and digitally signed by Alice's tax advisor, Alice only needs to prove to the bank that (a) the income is above the required threshold and (b) that the income was signed off as correct by the tax advisor.
The questions follow immediately:
How can the bank, the data consumer, specify which attested information they need to receive from Alice, the data holder?
And, how can then the data holder avoid disclosing any information beyond the required while proving the integrity it?

\begin{figure*}[t!]
    \centering
    \includegraphics[width=0.775\linewidth]{img/SSI-ZKP-ESWC-2026.png}
    \caption{The problem: Upon request, prove required data but nothing beyond.
    }
    \Description{A diagram illustrating the process of a data holder (Alice) deriving a presentation with proof from a digitally signed tax filing, in response to a bank's request for proof of income (income > 50K). The tax advisor issues the digitally signed tax filings, and the bank verifies the proof. The dashed lines indicate trust relationships and requests.}
    \label{fig:example}
\end{figure*}


To strike such a balance, Europe's leading cryptographers strongly advocate for the application of Zero-Knowledge Proofs (ZKPs) on top of\todo{JW: alongside} digital signatures~\cite{cryptoFeedback}.
Using a ZKP, a prover is able to convince a verifier that a claim is true without the verifier learning any additional information~\cite{DBLP:conf/stoc/GoldwasserMR85}.
To model asserted data and cryptographic provenance information, the W3C Verifiable Credentials (VCs)~\cite{VC} recommends a graph-based data model based on the Resource Description Framework (RDF)~\cite{RDF}.
Previous efforts on Linked Data Integrity Proofs~\cite{DataIntegrityBBS} were recently complemented by a definition of RDF-based semantics for selective disclosure~\cite{DBLP:conf/esws/BraunK25}.
It is logically sound to apply querying and reasoning techniques on selectively disclosing RDF datasets~\cite{DBLP:conf/esws/BraunK25}, \eg VCs and their presentations, even when applying ZKPs.
This logical consistency~\cite{DBLP:conf/esws/BraunK25} enables the central idea of our approach: using SPARQL~\cite{SPARQL}, the standard query language for RDF, to express which data or properties thereof to disclose.

In this paper, we introduce \textit{zkRDF}, a data-centric approach for selectively disclosing RDF datasets based on SPARQL-derived Zero-Knowledge Proofs.
Upon receiving a SPARQL query from a data consumer, a data holder executes the received query and discloses -- not the results -- but a subset of the queried dataset with attached proofs about the desired properties as expressed by the query.
Such a presentation of the queried dataset including corresponding proofs is called a \enquote{selectively disclosing dataset}~\cite{DBLP:conf/esws/BraunK25}.
So, rather than the data holder creating a proof that the query was executed correctly, the verifier receives a selectively disclosing dataset from which they are able to obtain the query results themselves.
% Both approaches result in the same data disclosure - and are thus equally privacy preserving.
There is no monolithic proof about SPARQL solution mappings; the produced proof is composed of sub-proofs about the underlying RDF dataset. 

Our evaluation shows that most SPARQL features are supported in \textit{zkRDF} while the remaining cannot directly be supported without privacy caveats.
In terms of performance, we compared our proof-of-concept (PoC) implementation\footnote{\url{https://anonymous.4open.science/r/rdf-zkp-sparql}} to a prototype following an alternative but comparable approach.
Their approach relies on a zero-knowledge virtual machine and is shown to be three orders of magnitude slower on their own benchmark than our approach.% We acknowledge that other, more performant, options for proving properties of computations are available - such as using circuit-builders.

%
We highlight our contributions:
\begin{itemize}[noitemsep,topsep=0pt,parsep=0pt,partopsep=0pt,wide,labelwidth=!,labelindent=0pt,leftmargin=14pt]
    \item a first distinction between data-centric and computation-centric\todo{JW: I would instead phrase this as "ZKP of query evaluation" vs "ZKP of data properties"} approaches to proving query results in RDF datasets; based on related work (Section~\ref{sec:related}) 
    \item \textit{zkRDF} (Section~\ref{sec:framework}): abstract methodology and conceptual instantiation for SPARQL-derived selective disclosure of RDF datasets (a formalisation is available in Appendix~\ref{appendix:zkrdf}).
    \item its evaluation (Section~\ref{sec:eval}) along the three dimensions of
    \begin{itemize}[noitemsep,topsep=0pt,parsep=0pt,partopsep=0pt,wide,labelwidth=!,labelindent=0pt,leftmargin=14pt]
        \item competence -- in terms of SPARQL feature support
        \item performance -- to compare implementations to alternatives
        \item privacy and security -- to highlight potential limitations
    \end{itemize}
    providing an initial evaluation framework.
\end{itemize}
To cover the necessary foundations, we first present preliminaries (Section~\ref{sec:prelim}). 
A brief example illustrates the approach from a practical perspective (Section~\ref{sec:example}).
After presentation of the contributions, we conclude the paper with a vision for future research (Section~\ref{sec:conclusion}).







\begin{comment}
On Proof of Completeness (The Prover's Choice)

fundamental feature is holder-centric privacy.
The goal is for the prover to prove they meet a certain requirement, the system not designed to prove that the holder has revealed all possible matching data.
\end{comment}



\section{Preliminaires}
\label{sec:prelim}
We briefly cover the fundamental concepts of SPARQL which the {zkRDF} approach uses to achieve selective disclosure of RDF datasets.

\paragraph{RDF~\cite{RDF},} the Resource Description Framework, provides a graph-based data model. 
We re-use the common formalization of RDF datasets from~\cite{DBLP:conf/esws/BraunK25}, provided Appendix~\ref{appendix:rdf} for completeness.
We agree to the choice of semantics for RDF datasets from~\cite{DBLP:conf/esws/BraunK25}:
When merging two RDF datasets, blank nodes between the two RDF datasets must be re-labeled to avoid co-references, and new graph names must be minted for the default graphs of the respective RDF datasets.
% \todo{JW: suggesst replacing with: "Per RDF semantics [insert link] blank nodes are not co-referencable across graphs" and then mention the re-naming as an implementation detail later on}

The Verifiable Credentials (VCs) data model~\cite{VC}, a W3C Recommendation, models assertions and cryptographically assured provenance, \eg digital signatures, as an RDF dataset.
In that RDF dataset, the claims and credential metadata form the unnamed default graph, and a particular claim links to a named graph with corresponding cryptographic provenance information, \ie the proof graph.
\begin{comment}
\todo{JW: Suggest removing everything after this - as it is an opinion/position and draws focus from the paper}Based on our implementation experience, briefly outlined in Section~\ref{sec:eval_performance}, we agree with~\cite{DBLP:conf/esws/BraunK25} that the W3C VC data model should mandate asserting claims in a named graph, and have the proof link to the graph it covers.
\end{comment}
Nanopublications~\cite{DBLP:journals/peerj-cs/KuhnCKQVGNVD16} offer an alternative RDF-based data model where provenance information is modeled in a named graph and linked to a named graph of assertions.

\paragraph{SPARQL~\cite{SPARQL}} is the W3C-recommended query language for RDF. %\todo{check paragraph}
The fundamental component of a SPARQL query is the {Basic Graph Pattern (BGP)}, a set of triple patterns. 
The evaluation of a BGP against an RDF graph yields a multiset of {solution mappings}
Each solution mapping maps variables to RDF terms.
The semantics of complex SPARQL queries are formally defined by an algebra over these multisets of solution mappings. 
One of the key algebraic operators is {\texttt{Filter}}. 
A filter operation reduces the multiset of solution mappings based on an {expression}. %\todo{define expression better?}
Such expressions are constructed from variables, RDF literals, and a set of built-in functions, including logical connectives (\texttt{\&\&}, \texttt{||}), relational comparisons (\texttt{=}, \texttt{<}, \texttt{>}), and type tests (e.g., \texttt{isLiteral()}). % \todo{JW: Consider linking to the point in the spec defining expressions}

\begin{comment}
SPARQL may thus serve as one way to query and filter information with attached cryptographic provenance information.
For example, a SPARQL query may express which VCs should be presented to a verifier and which information therein should be revealed or remain hidden. \todo{JW: Suggest removing this papragaph on filtering cryptographic information - to avoid distraction / confuison}
\end{comment}


\paragraph{Selective disclosure} refers to the concept of proving properties of a dataset while hiding some or all of the data~\cite{Brands2002ATO}.
One way to implement this selective disclosure are zero-knowledge proofs (ZKPs)~\cite{DBLP:conf/stoc/GoldwasserMR85}: A prover is able to convince a verifier that a claim is true without the verifier learning any additional information (zero-knowledge).
An example of a claim is: \texttt{My secret age \textit{w} is greater than 21}.
Terminology-wise, a {claim} consists of the following elements~\cite{DBLP:journals/iacr/CampanelliFQ19}:
What is to be proven is called \textbf{relation}, \ie \texttt{\textit{secret value} greater than \textit{public value}}.
The set of public inputs to the relation is called (proof) \textbf{statement}, \ie \texttt{21}.
The set of private inputs, secret to the prover, is called \textbf{witness}, \eg \texttt{25}.

ZKPs can be used to prove a number of different relations;
proving numeric bounds~\cite{DBLP:journals/iacr/CampanelliFQ19, DBLP:journals/iacr/Eagen22} or set (non-)membership~\cite{DBLP:journals/iacr/VittoB20}.
Or, most common in the domain of VCs is proving knowledge of signature~\cite{DBLP:conf/scn/CamenischL02, DBLP:journals/iacr/CamenischDL16, cryptoeprint:2015/525} on credential data while simultaneous revealing some data in the credential and hiding the remainder.
Proof composition, \ie combinations of ZKPs on the same witness data, is enabled by following the {commit-and-prove scheme}~\cite{DBLP:journals/iacr/CanettiLOS02,DBLP:books/daglib/0066918}.
% \todo{JW: I would move discussion of different ways of performing proof of knowledge of signature to a later section}

\begin{comment}
\todo{JW: I don't think the following 3 paragraphs about comitting to a dataset add to this section. I would argue that it is a fair assumption that data is immutable after sigining. I would delete them.}
When an RDF dataset is digitally signed by an issuer, the data is being committed to \todo{JW: Define "comitted to"}.
The corresponding signature is only valid for the current state of the dataset; the underlying data cannot be changed after the fact.
Then, a prover is able to create a proof, \eg a proof of knowledge of signature or of numeric bounds, about the dataset.
\end{comment}


\section{Related Work}
\label{sec:related}

We acknowledge that provenance surrounding SPARQL has been a long researched field, and yet, to our knowledge, no related work at the intersection of SPARQL and ZKPs exists -- except for one project\footnote{\label{fn:zkVM}\url{https://github.com/jeswr/queryable-credentials}} found on GitHub.
This project takes a computation-centric perspective on query execution, similar to related research on SQL and Cypher.
Taking a data-centric perspective, we examine the intersection of RDF and ZKPs, most relevant for VCs~\cite{VC}.


\paragraph{Computation-centric approaches} to zero-knowledge querying focus on proving the correctness of a result that is computed or aggregated from a dataset. The primary goal is to provide a verifiable answer to an (often analytical) query without revealing the underlying individual data that was used in the computation.

For SPARQL, an explorative work on queryable credentials\footnote{See footnote \ref{fn:zkVM}.} uses a zero-knowledge Virtual Machine (zkVM) to prove the execution and thus results of a query.
% A zkVM is designed to prove the correct execution of arbitrary input programs; one such program can be a SPARQL engine. 
In this SPARQL-in-zkVM approach, the zkVM proves that a SPARQL query engine executed the input SPARQL query correctly.
The query execution is handled within the zkVM and a corresponding execution trace is created as a proof.
In the use case of this work, the dataset includes digitally signed RDF graphs, i.e. VCs.
The program executed within the zkVM also includes a verification of those VCs to prove that the SPARQL results were obtained from digitally signed RDF graphs.
% A check on which issuer signed which data underlying the query results is not (yet) supported in this prototype.

For SQL, the ZKSQL approach~\cite{DBLP:journals/pvldb/LiWXWR23}
aims to prove correct execution of SQL queries on relational databases.
This approach decomposes a query based on the standard relational algebra query plan into SQL operators like \texttt{join}, \texttt{filter}, and \texttt{aggregate}.
It then relies on an interactive ZK proof protocol that requires real-time, back-and-forth communication between the prover and verifier to generate and validate a proof of correct SQL query execution.

% \todo{JW: cite Pponeglyphdb, the writeup on P29 of https://jeswr.solidcommunity.net/public/tos-report-v4-cleaned-2.pdf can be used}

For Cypher, the ZKGraph approach~\cite{DBLP:journals/corr/abs-2507-00427} targets graph databases to prove complex algorithmic operations like pathfinding and multi-hop traversals.
This approach employs an \enquote{expansion-centric} model where a query is broken down to the core primitive, \ie graph traversal (node expansion), and other operations are treated as variations (that have certain \enquote{attributes}) of this primitive.
The prover then generates a single, self-contained proof that the query results had been correctly computed.

\paragraph{Data-centric approaches} to zero-knowledge querying focus on proving that a specific subset of data is authentic and satisfies a set of declarative properties. The primary goal is the selective and verifiable disclosure of data items themselves or properties of them, not a new value computed from them.

A first approach to ZKPs on RDF-based credentials~\cite{DBLP:conf/eurosp/YamamotoSS22} only allows selective disclosure of complete triples of an RDF graph.
Proofs on individual RDF terms of a triple are not possible.
The current Data Integrity BBS Cryptosuites~\cite{w3c-vc-di-bbs-2025}, a Candidate Recommendation Draft by the W3C Verifiable Credentials Working Group, follows a similar approach.

Considering RDF-based semantics~\cite{DBLP:conf/esws/BraunK25} provides a more granular approach for selective disclosure: This approach allows to selectively disclose RDF terms within a quad of
an RDF dataset, \eg to prove numeric bounds or set (non-)membership on individual RDF terms; or equality of RDF terms across quads.
Moreover, in this approach, it is proven to be logically sound to
query and to reason on VCs and their presentations, even when using ZKPs.
This paper builds on the semantic foundations provided in~\cite{DBLP:conf/esws/BraunK25} and extends it by enabling definition of what to prove using SPARQL.


In the particular domain of digital credentials, the standard for querying is the Digital Credentials Query Language (DCQL) -- not SPARQL.
DCQL is a JSON-based language defined by the recent OpenID4VP standard~\cite{openID4VP}.
It aims to provide a credential-format independent way to express which credential(s) or data therein should be presented to a verifier.
DCQL is deliberately limited: A verifier can only ask for credentials in a restricted way to prevent data leakage and to protect the user.
From the current specification, it seems that only queries for particular credentials, credential formats, and attested attributes can be specified.
DCQL thus does not seem to support definition of numeric bounds or set (non-)membership.
The approach presented in this paper, however, is not limited to the use case (and data model) of VCs, but it is a rather general approach to prove correctness of SPARQL query results from digitally signed RDF datasets using ZKPs.

\section{Illustrating Example}
\label{sec:example}

In this section, we illustrate the data-centric nature of zkRDF to aid presenting the details of the approach in Section~\ref{sec:framework}.

Consider again our initial example from Figure~\ref{fig:example}.
Freelance consultant Alice is issued their tax filings by their tax consultant in form of a digitally signed dataset, \eg, Verifiable Credentials containing the tax filing information modeled in Listing~\ref{listing:run-example-A} for 2024 and Listing~\ref{listing:run-example-B} for 2023.
\begin{table}[!b]
\centering
\begin{tabular}{p{0.46\linewidth} p{0.46\linewidth}}

% Left Column: Listing for Fiscal Year 2024
\lstset{style=turtle}
\begin{lstlisting}[
    language=Turtle, 
    caption={Simplified tax filing (2024); Turtle syntax.},
    label={listing:run-example-A},
    basicstyle=\scriptsize\ttfamily
]
[] a fin:TaxReturn ;
 fin:about ex:alice ;
 fin:provider ex:trustedTaxAdvisor;
 fin:taxPeriod "2024" ;
 fin:taxableIncome [
  a fin:MonetaryAmount ;
  fin:currency "USD" ;
  fin:value "123000"^^xsd:integer
 ] . # signature details omitted
\end{lstlisting}

& % Column separator

% Right Column: Listing for Fiscal Year 2023
\lstset{style=turtle}
\begin{lstlisting}[
    language=Turtle, 
    caption={Simplified tax filing (2023); Turtle syntax.},
    label={listing:run-example-B},
    basicstyle=\scriptsize\ttfamily
]
[] a fin:TaxReturn ;
 fin:about ex:alice ;
 fin:provider ex:trustedTaxAdvisor;
 fin:taxPeriod "2023" ;
 fin:taxableIncome [
  a fin:MonetaryAmount ;
  fin:currency "USD" ;
 fin:value "90000"^^xsd:integer
 ] . # signature details omitted
\end{lstlisting}
\\
\end{tabular}
\end{table}

When Alice asks a bank for a loan, the bank requires certain information to prepare a suitable loan offer.
For example, the bank requires that last year's income is above 85000 (USD).
To receive an offer with a more beneficial interest rate, the bank optionally requests proof that the income in the year before was also above the threshold.
For a first loan offer, the bank does not require a user's identifier -- only financial information are relevant in this stage of the process.
We express such a query using SPARQL (Listing~\ref{listing:run-example-Q}).

\begin{table}[!t]
\centering
\begin{tabular}{p{0.96\linewidth} }
\lstset{style=sparql}
\begin{lstlisting}[
    language=SPARQL, 
    caption={The bank's SPARQL query for an income greater than 85000 (USD) in 2024, and optionally in 2023.} ,
    label={listing:run-example-Q}, 
    basicstyle=\scriptsize\ttfamily
]
SELECT ?taxAdvisor # disclose which tax advisor signed the filing
WHERE {
    BIND(85000 as ?requiredIncome)
    ?return2024     a     fin:TaxReturn ;
        fin:about         ?user ;  # initially, hide user (not projected)
        fin:provider      ?taxAdvisor ; # find tax advisor 
        fin:taxPeriod     "2024" ;
        fin:taxableIncome [ fin:currency "USD" ;
                            fin:value ?income2024 ] .
    FILTER(?income2024 > ?requiredIncome) # check income threshold
    # optionally, to get a better rate provide more information
    OPTIONAL {
    ?return2023     a     fin:TaxReturn ;
        fin:about         ?user ; # require filing for same user
        fin:provider      ?taxAdvisor ; # find tax advisor
        fin:taxPeriod     "2023" ;
        fin:taxableIncome [ fin:currency "USD" ;
                            fin:value ?income2023 ] .
    FILTER(?income2023 > ?requiredIncome)  # check income threshold
}   }

\end{lstlisting}
\\
\end{tabular}
% \end{table}

% \begin{table}[!t]
% \centering
\begin{tabular}{p{0.96\linewidth} }
\lstset{style=turtle}
\begin{lstlisting}[
    language=Turtle, 
    caption={An excerpt of the selectively disclosing RDF dataset that is presented to the bank, \ie the data consumer and verifier. Cryptographic details are omitted for brevity.} ,
    label={listing:run-example-P}, 
    basicstyle=\scriptsize\ttfamily
]
GRAPH _:0_4 {
	_:0_0 a fin:TaxReturn .
	_:0_0 fin:about _:0_7 .
	_:0_0 fin:provider ex:trustedTaxAdvisor .
	_:0_0 fin:taxPeriod "2024" .
	_:0_0 fin:taxableIncome _:0_22 .
    _:0_25 _:0_26 _:0_27 . 
    _:0_22 fin:currency "USD".
    _:0_22 fin:value _:0_37 .
}
GRAPH _:2_4 {
	_:2_0 a fin:TaxReturn .
	_:2_0 fin:about _:0_7 .
	_:2_0 fin:provider ex:trustedTaxAdvisor .
	_:2_0 fin:taxPeriod "2023" .
	_:2_0 fin:taxableIncome _:2_22 .
    _:2_25 _:2_26 _:2_27 . 
    _:2_22 fin:currency "USD".
    _:2_22 fin:value _:2_37 .
}
GRAPH _:presentationProofGraph {
	_:cproof rdf:type zkp:CompositeProof .
	_:cproof zkp:hasComponent _:p0 , _:p1 , _:p2 , _:p3 .
    _:0_7 spok:hasSchnorrResponse "a..f"^^xsd:base64Binary.# ex:alice proven
	_:p0 rdf:type bbsp16:PoKS16 . # proof of knowledge of signature
    _:p0 bbsp16:isProofOfKnowledgeOfSignatureOverGraph _:0_4 .
	_:p1 rdf:type lg16:LegoGroth16ProofOfRangeMembership . # numeric bounds
    _:p1 lg16:hasWitness _:0_37 .
	_:p1 lg16:hasLowerBound "85000"^^xsd:nonNegativeInteger .
	_:p2 rdf:type bbsp16:PoKS16 . # proof of knowledge of signature
	_:p2 bbsp16:isProofOfKnowledgeOfSignatureOverGraph _:2_4 .
    _:p3 rdf:type lg16:LegoGroth16ProofOfRangeMembership . # numeric bounds
    _:p3 lg16:hasWitness _:2_37 .
	_:p3 lg16:hasLowerBound "85000"^^xsd:nonNegativeInteger .
	# more proof details omitted for brevity
}
\end{lstlisting}
\\
\end{tabular}
\end{table}

Upon receiving this query, Alice -- or rather their data management system, \eg their EUDI wallet~\cite{euIDwallet} or Solid Pod~\cite{SolidProtocol} -- searches the available RDF dataset for resulting query results.
A solution is found where the optional information on the 2023 tax filings are included.
To get a better interest rate on their loan, Alice decides to keep these optional bindings in the results.

A selectively disclosing dataset is thus created.
Listing~\ref{listing:run-example-P} shows the RDF dataset to be presented to the bank:
It is comprised of RDF graphs where some information remains hidden -- because it is not required to be disclosed.
Blank nodes serve as stand-ins, and corresponding proofs of knowledge of the underlying original values are attached.
While Alice's identifier remains hidden (see blank node \texttt{\_:0\_7}), the identifier is still proven to occur in both tax filings.
In addition, proofs of knowledge of signatures that show that the revealed and hidden information, \ie the tax filing, had been signed by a particular and known tax advisor.
Finally, proofs of numeric bounds (cf. L27-30, L33-35) show that the attested income from 2024 and from 2023 are indeed above the required threshold.

For the cryptographic foundations, we refer to~\cite{DBLP:conf/esws/BraunK25}. 
We highlight in the context of Listing~\ref{listing:run-example-P} that the term \texttt{\_:0\_7} is re-used across graphs \texttt{\_:0\_4} and \texttt{\_:2\_4}. 
The proof includes the fact that all occurrences of \texttt{\_:0\_7} refers to the same underlying secret value, Alice's identifier. 
Similarly, for the proofs of numeric bounds of \texttt{\_:0\_37} and \texttt{\_:2\_37} indicate that the particular values that were indeed signed by the tax advisor -- proven by the proof of knowledge of signatures over graphs \texttt{\_:0\_4} and \texttt{\_:2\_4} respectively.
Note that the tax advisor's identifier is revealed, as requested in the \texttt{SELECT} statement of the query.

All other information are only proven to be true:
\begin{itemize}[noitemsep,topsep=0pt,parsep=0pt,partopsep=0pt,wide,labelwidth=!,labelindent=0pt,leftmargin=14pt]
\item[(a)] The signatures on the graphs are proven to be valid, the graphs' triples are proven to be known, without needing to reveal terms.
\item[(b)] The relations between Alice and their tax filing are proven to exist without revealing their identifier.
\item[(c)] The numeric bounds as per \texttt{FILTER} statements are proven to be fulfilled without revealing the literals.
\end{itemize}

Only information explicitly requested (or previously known as the URIs) in the query are revealed, while still being proven to have been signed. For example in graph \texttt{\_:0\_4}, we do not reveal that \texttt{\_:0\_25} actually the same value that also underlies \texttt{\_:0\_22}.

\section{The zkRDF Approach}
\label{sec:framework}

In this section, we introduce \textit{zkRDF}, a data-centric approach for selectively disclosing SPARQL query results from RDF datasets using ZKPs.
We first present the common abstract methodology (Section~\ref{sec:framework_abstract}) that underlies its conceptual instantiation (Section~\ref{sec:framework_instantiation}) which is a direct extension of the work presented in \cite{DBLP:conf/esws/BraunK25}.
We provide an initial formalisation in Appendix~\ref{appendix:zkrdf}.
A proof-of-concept implementation is available online\footnote{\url{https://anonymous.4open.science/r/rdf-zkp-sparql}}.

\subsection{Abstract Methodology}
\label{sec:framework_abstract}

\begin{figure*}[!t]
    \centering
    \includegraphics[width=0.99\linewidth]{img/zk-SPARQL-horizontal.png}
    \caption{An illustration of zkRDF's abstract methodology to achieve query-derived selective disclosure. Rounded nodes represent data, gray boxes represent procedures. Edge labels qualify the link between data and a procedure.}
    \Description{Diagram showing zkRDF methodology with rounded nodes for data, gray boxes for procedures, and labeled edges.}
    \label{fig:methodology}
\end{figure*}
The central idea of the zkRDF approach is the systematic transformation of a SPARQL query and its solution into a set of relations, statements, and witnesses to be proven.
This approach, illustrated in Figure~\ref{fig:methodology}, consists of the following procedures:

\begin{enumerate}[noitemsep,topsep=0pt,parsep=0pt,partopsep=0pt,wide,labelwidth=!,labelindent=0pt,leftmargin=14pt]   
    \item[\textbf{Query Execution.}] The query engine executes a query and produces a set of solution mappings.
    
    \item[\textbf{Query Analysis.}] The query is dissected and relations, statements and the selection of witnesses are derived:
    \begin{itemize}[noitemsep,topsep=0pt,parsep=0pt,partopsep=0pt,wide,labelwidth=!,labelindent=0pt,leftmargin=14pt]
        \item[Solution Modifiers] specify which variable mappings are to be projected.
        In the solution analysis, it is thus determined which variable mappings should made public and which should remain private.
        \item[Graph Patterns] specify the structure of the data to be matched, \ie what data is relevant and where relationships must exist.
        Graph Patterns thus entail relations and corresponding statements for proofs of knowledge of signatures on data matching the respective patterns.
        \item[Expressions] specify logical and arithmetic conditions that the matched data must satisfy. This determines how solutions are validated or constrained.
        Expressions thus entail relations and corresponding statements for \eg proofs of numeric bounds.
    \end{itemize}
    
    \item[ \textbf{Solution Analysis.}] Based on the selected variables to project and the solution mappings, it is determined which RDF terms are made public, \ie are part of the statements, and which remain private, \ie are witnesses.

    \item[\textbf{Proof Creation.}] The proof system creates a proof from the set of relations, statements, and corresponding witnesses.
    Moreover, the system provides a serialization of the composite proof to be presented to a verifier.
\end{enumerate}





\subsection{Conceptual Instantiation}
\label{sec:framework_instantiation}

We present one approach to instantiate the abstract methodology for selectively disclosing an RDF dataset using SPARQL-derived ZKPs.
Our approach is based on both the semantics and the techniques presented in~\cite{DBLP:conf/esws/BraunK25}.
In this approach,  
the results of an input SPARQL query are indirectly proven to be cryptographically correct via selective disclosure of the queried dataset.

That is, in contrast to the computation-centric approaches, which produce a proof of solution mappings, this data-centric approach provides proofs about the underlying dataset such that the verifier can obtain the query results from the selectively disclosed dataset.
The presented dataset thus needs to include cryptographically verifiable data for each solution mapping to be obtained by the verifier. 

\paragraph{On the data holder's (prover) side.}
%
To produce such a selectively disclosing dataset, 
we need determine for each solution mapping:
\begin{itemize}[noitemsep,topsep=0pt,parsep=0pt,partopsep=0pt,wide,labelwidth=!,labelindent=0pt,leftmargin=14pt]
\item[(a)] which RDF terms in which quads of the dataset are part of that solution mapping and should thus be revealed, 
\item[(b)] which RDF terms in which quads of the dataset helped produce that solution mapping by matching the query's graph pattern but are kept hidden, and 
\item[(c)] which of the hidden RDF terms fulfill \texttt{FILTER} expressions from the query that need to be proven.
\end{itemize}
This requires tracing the \enquote{origins} of a solution mapping, which includes tracing all non-disclosed variables and blank nodes in the graph patterns.
This particular approach is comprised of the following steps; formalised in Appendix~\ref{appendix:zkrdf}:
\begin{enumerate}[noitemsep,topsep=0pt,parsep=0pt,partopsep=0pt,wide,labelwidth=!,labelindent=0pt,leftmargin=14pt]
\begin{comment}
    \item[\textit{1. Rewrite Query:}]
    Any blank node in a graph pattern is replaced with a fresh and unique variable. 
    Additionally, all variables are to be projected.
    This is to trace all variables involved in the query logic, not only the ones the verifier is interested in.\todo{JW: remove this step - it is just implementing SPARQL semantics}
\end{comment}

    \item[\textit{1. Execute Query:}]
     The query is executed against the RDF dataset to obtain the set of solutions mappings.

    \item[\textit{2. Trace Each Mapping (before projection):}]
    For each solution mapping, trace every mapped RDF term back to its occurrence in the underlying RDF dataset. 
    This includes intermediate variables, \ie non-projected variables and all blank nodes in the query's graph patterns.

    \item[\textit{3. Derive Relations, Statements and Witnesses:}]~
    In the scope of this work, we derive from the query, its solution and its trace:
    \begin{itemize}[noitemsep,topsep=0pt,parsep=0pt,partopsep=0pt,wide,labelwidth=!,labelindent=0pt,leftmargin=14pt]
        \item[{Proofs of Knowledge of Signature.}]
        For each solution mapping, for each of the datasets graphs that are relevant to the SPARQL query, a corresponding proof of knowledge of signature on that graph is to be added to the proof composition.
    
        \item[{Proofs of Numeric Bounds.}]
        Inspect the query's \texttt{FILTER} expressions to extract relational expressions that indicate numeric constraints.
        If the variable in the filter expression is not projected, a corresponding proof of numeric bounds is to be added to the proof composition.

        \item[{Equality Constraints.}]
        When a particular variable-mapped RDF term remains hidden, equivalence of the hidden term across its occurrences relevant to the SPARQL query is proven.
    \end{itemize}
    For each relation, determine which RDF terms should be revealed as part of the proof statement and which should be kept hidden as witnesses.
    Terms mapped to variables to be projected are to be revealed.
    All other mapped terms are to be hidden.
    Constants from the query are revealed; they are already known to the verifier who formulated the query.
    
    \item[\textit{4. Create Proof and Presentation:}]
    Given the relations, statements and witnesses create a composite proof and generate a corresponding presentation of the selectively disclosed dataset.
\end{enumerate}

\noindent
The presented approach instantiates the abstract methodology as follows:
Query execution is in direct correspondence; the query is executed \eg using a regular SPARQL query engine.
Tracing each solution mapping is conceptually part of the solution analysis procedure.
Technically, tracing may be implemented during query execution. %; \eg as suggested by our formalization provided in Appendix~\ref{appendix:zkrdf}.
Deriving relations, statements and witnesses instantiates parts of both query analysis and solution analysis. 
Separation of concern between the two analysis procedures is not completely clear cut in this instance.
Finally, creating the proof and presentation is in direct correspondence.% \todo{I don't get this section mapping to the abstract methodology - I suggest remove or clarify}

In Appendix~\ref{appendix:zkrdf}, we provide a more formal description this instantiation based on extending the SPARQL query evaluation to directly provide the required tracing information during query execution.

\paragraph{On the verifier's (consumer) side.}
%
The verifier receives the presentation from the prover.
To obtain query results, the verifier executes the SPARQL query or a re-written version of it on the presentation.

In case of a simple SPARQL query that does not have a FILTER statement for numeric bounds, the verifier can directly execute the query:
For each solution mapping that the prover obtained from the original dataset, the individual graphs in the selectively disclosing dataset are entailed from the underlying respective graphs in original dataset under simple entailment~\cite{DBLP:conf/esws/BraunK25}.
This means that the graph patterns that the SPARQL query defines are preserved and thus will find the corresponding matches as if executed on the original dataset.

In case of a SPARQL query that includes \texttt{FILTER} statements for numeric bounds, we need to rewrite the query if the variables used in the \texttt{FILTER} statements are not projected.
If they were projected, then the corresponding values are revealed and the verifier can check the \texttt{FILTER} statement themselves.
If they are indeed not projected, then the \texttt{FILTER} statement will filter out the blank node that acts as a stand-in for the hidden literal.
In this case, the \texttt{FILTER} statement is replaced by a graph pattern that specifies a proof of numeric bounds where the bounds correspond to what was specified in the \texttt{FILTER} and the witness is specified using the variable from the \texttt{FILTER}\todo{JW: This needs wordsmithing to improve clarity}.
At this point, it is sufficient for the SPARQL query to only check for existence of a proof of numeric bounds which proves that the property specified in the \texttt{FILTER} holds.
Verifying that the property is indeed proven, \ie that the \texttt{FILTER} was indeed correctly applied, is part of the validity check~\cite{DBLP:conf/esws/BraunK25} on the selectively disclosing dataset.
%\todo{JW: Do you define what a validity check is in this paper - if not spend 1-2 sentences doing so}.











\begin{comment}
We summarize the main aspects of this approach:
\begin{itemize}
\item[{Goal:}] Prove that a private dataset contains solution mappings to a query.
\item[{Components:}]  Query engine, query-to-claim compiler, composite proof system
\item[{Inputs:}]      SPARQL query, RDF dataset incl. issuers' public keys and signatures
\item[{Process:}]    Rewrite and execute query, compile claims from analysis of query and solution, create proofs about the dataset to prove correctness of solution
\item[{Output:}]    Selectively disclosed dataset incl. composite proof
\end{itemize}
\end{comment}


\begin{comment}
    graph TD

%% Define styles for clarity
%% Data: Unfilled (white), stadium shape (highly rounded)
%% Process: Filled (gray), sharp rectangle
classDef data fill:#fff,stroke:#333,stroke-width:2px;
classDef process fill:#ddd,stroke:#333,stroke-width:2px;
classDef none fill:none,stroke:none;

   
    subgraph inputs [ ]
        Q([SPARQL Query])
        DS([Private Dataset])
    end

    subgraph querying [ ]
        QE["Query Engine"]
        Q -->|"is input to"| QE
        DS -->|"is input to"| QE
        SM([Solution Mappings])
        QE -->|"produces"| SM
    end
        
    subgraph analysis [ ]
        QA["Query Analysis"]
        Q -->|"is input to"| QA 
        QP([Selected Variables])
        QA -->|"extracts"| QP
        QP -->|"is input to"| SA

        SA["Solution Analysis"]
        SM -->|"is input to"| SA

        R(["Relations<br/>(to prove)"])
        S(["Statements<br/>(public inputs)"])
        W(["Witnesses<br/>(secret inputs)"])   
        QA -->|"derives"| R
        QA -->|"extracts"| S
        SA -->|"reveales"| S
        SA -->|"hides"| W
    end

subgraph proves [ ]
PS["Proof System"]
    R -->|"is input to"| PS
    S -->|"is input to"| PS
    W -->|"is input to"| PS

    P(["Composite Proof<br/>& Presentation"])
    PS -->|creates| P
end

%% Apply the styles
class Q,QP,DS,R,S,W,SM,P data;
class QA,QE,PS,SA process;
class inputs,proves none;
 class querying,analysis none;

\end{comment}


\begin{comment}

It is thus the question which particular proofs about the underlying dataset need to be composed such that a verifier would obtain correct query results.
To provide an intuition:
The query in Listing~\ref{listing:run-example-query}, a simple example, specifies a graph pattern resembling a membership card and a FILTER statement asking that the membership is not yet expired.
Moreover, the organization that the membership is attested of is to be projected; the membership's user or its end date do not need to be disclosed.
\lstset{style=sparql}
\begin{lstlisting}[
    language=SPARQL, 
    caption={A simple query to illustrated how properties to be proven are derived.},
    label={listing:run-example-query}, 
    basicstyle=\scriptsize\ttfamily,
]
SELECT ?org WHERE {
  ?membership org:member ?user.
  ?membership org:organization ?org .
  ?membership time:hasEnd ?endDate .
  FILTER(?endDate > xsd:dateTime("2025-10-01T00:00:00Z") )
}
\end{lstlisting}
A solution to this query would then entails a corresponding proof that the triples matching the graph pattern had indeed been signed by an issuer, \eg the organisation itself, and a proof of numeric bounds that the end date, which remains hidden, is indeed in the future.
Listing~\ref{listing:run-example-present} provides an example of a resulting selectively disclosing dataset.
Note \eg that blank node \texttt{\_:0\_12} is both object of the triple in line 4 and in line 16, meaning that the underlying literal is proven to be  part of the membership card and that the same value is proven to be within numeric bounds.
\lstset{style=turtle}
\begin{lstlisting}[
    language=Turtle, 
    caption={result P} ,
    label={listing:run-example-present}, 
    basicstyle=\scriptsize\ttfamily
]
GRAPH _:0_4 {
	_:0_0 _:0_1 _:0_2 .
	_:0_5 _:0_6 _:0_7 .
	_:0_10 time:hasEnd _:0_12 .
	_:0_10 org:member _:0_17 .
	_:0_10 org:organization <http://example.org/organisations#aCompany> .
}
GRAPH _:presentationProofGraph {
	_:cproof rdf:type zkp:CompositeProof .
	_:cproof zkp:hasComponent _:p0 .
	_:cproof zkp:hasComponent _:p1 .
	_:p0 rdf:type bbsp16:PoKS16 . # proof of knowledge of signature
    _:p0 bbsp16:isProofOfKnowledgeOfSignatureOverGraph _:0_4 .
    # more proof details omitted for brevity
	_:p1 rdf:type lg16:LegoGroth16ProofOfRangeMembership . # numeric bounds proof
    _:p1 lg16:hasWitness _:0_12 .
	_:p1 lg16:hasLowerBound "1759276800"^^xsd:nonNegativeInteger .
	_:p1 lg16:hasUpperBound "18446744073709551615"^^xsd:nonNegativeInteger .
	# more proof details omitted for brevity
}
\end{lstlisting}
\end{comment}
\section{Evaluation}
\label{sec:eval}

We evaluate our zkRDF approach by assessing its competence in terms of which SPARQL features~\cite{SPARQL} are or could be supported.
In addition, we compare performance between our proof-of-concept implementation\footnote{\url{https://anonymous.4open.science/r/rdf-zkp-sparql}} with the SPARQL-in-zkVM project\footnote{\url{https://github.com/jeswr/risc0-ed25519-zk-sparql}}, the implementation of the related work on queryable credentials\footnote{\url{https://github.com/jeswr/queryable-credentials}}.
Finally, we discuss security and privacy implications in this approach.
We see the evaluation along the three dimensions of competence, performance (or efficiency), and security \& privacy as a potential framework to aid evaluating alternative, comparable approaches.

\subsection{Competence: SPARQL Feature Support}
\label{sec:eval_competence}

\paragraph{Query Forms.} 
In zkRDF, the query expresses to a data holder (prover) what information the data consumer (verifier) desires to be proven.
For this purpose, SPARQL's different query forms are interpreted different from their specified behavior for query evaluation.
Instead of providing regular SPARQL query results, the approach selectively discloses and proves properties of the underlying  dataset. 

For a \texttt{SELECT} query, the approach does not yield a set of solution mappings but instead provides a selectively disclosing dataset from which the solution mapping can be retrieved.
Variables specified for projection are considered explicitly included to be revealed, while all other variables are implicitly excluded from disclosure and thus remain hidden.

For an \texttt{ASK} query, the approach does not yield a boolean value but instead provides a selectively disclosing dataset. The ASK query is interpreted as a SELECT query without any variables to project.
This means, if the ASK query results in \texttt{true}, then the selectively disclosing dataset proves that result to indeed exist.
A query result of \texttt{false} cannot be proven in the zkRDF approach:
A selectively disclosing dataset cannot prove non-existence of query results.
The underlying original dataset may contain quads matching the query's graph patterns and expressions but this fact may remain hidden in the final selectively disclosing dataset, \eg when the number of results is restricted using \texttt{LIMIT}.
In this case, such excluded query results remain hidden -- proving that they do not exist is not possible in this approach.

For a \texttt{CONSTRUCT} query, the approach behaves the same as for a SELECT query where all variables from construct clause are to be projected.
The result is a selectively disclosing dataset from which the query-defined graph can be constructed by the data consumer.

The \texttt{DESCRIBE} query form is not supported, as it ambiguous what data to selectively disclose as a result.


\paragraph{Graph Patterns.} 
The zkRDF approach supports almost all operators on graph patterns trivially.
The approach simply does not prove the application of these operators on a solution set, which would be a computation-centric proof, but rather proves that a match satisfies the constraints imposed by the operators and their graph patterns, which is a data-centric proof. 
Therefore, \texttt{JOIN}, \texttt{UNION}, \texttt{LATERAL}, 
%\todo{JW: Lateral is not an operator - do you mean OPTIONAL?}
\texttt{LEFTJOIN}, 
%\todo{JW: MINUS breaks the Open World Assumption; I assume that you are using a non-membership proof to handle this. In the security section I would discuss that operators violating the OWA assumption should be avoided in multi-credential use-cases, as dropping a credential from the input can be a way of working around the MINUS}
and \texttt{GRAPH} are all supported.

\texttt{LEFTJOIN} could additionally be treated by requiring a decision on whether or not to include the optionally queried data. 
This decision could be automated or delegated to a supervising user.
As the approach presented in~\cite{DBLP:conf/esws/BraunK25} operates on RDF datasets, the \texttt{GRAPH} clause is also directly supported.

Only \texttt{MINUS} is not supported: 
Although the application of the operator works as expected -- it reduces the set of solution mappings that are to be proven to exist --, the remaining results cannot be proven to satisfy the restriction, \ie, to not satisfy the graph pattern from the \texttt{MINUS} clause.
Similar to the \texttt{ASK} query form, proving non-existence of information is not possible here.
%\todo{JW: To my previous comment - if a MINUS exists in the query, do you prove that all results in the results set should not have been MINUS'd by performing a set non-membership check?}.



\paragraph{Expressions.}
In the scope of this work, the zkRDF approach focuses on relational expressions in \texttt{FILTER} statements to express numeric bounds.
Integers are natively supported while dates and time need special consideration (cf. Appendix C of~\cite{DBLP:conf/esws/BraunK25}).\todo{Be more explicit about what is and is not supported. Specifically: What datatypes are supported? Is coercion between different numeric types supported? Is numeric equality supported correctly - can you prove that 2:int = 2.0:double? Note that filter equality is not a SAMETERM check - this fooled me for quite a while! The best way to do this may be to enumerate the operations supported in the SPARQL binary operators table at \url{https://www.w3.org/TR/sparql11-query/}}
In terms of logical expressions, AND (\texttt{\&\&}) and OR (\texttt{||}) are supported for the supported numeric relational expressions.

The way that zkRDF, based on the transformation approaches presented in~\cite{DBLP:conf/esws/BraunK25}, currently encodes RDF terms does not support all string expressions in zero-knowledge, \ie, manipulating strings or inspecting single characters of a string\todo{JW: I suggest omitting discussion of what features are NOT supported and why (incliding for strings, conditional expressions and aggregates). The \textit{what} is implicit (everything that you've not explicitly stated supporting), the why in case of strings is not informative. CB: I dare to disagree. The \textit{what} provides grounds to talk about the \textit{why}, which in turn let's us draw conclusions and give guidance. Here, it provides the basis to state that encoding terms is an important consideration for this type of work.}.
If such features were to be supported, a different way of transforming RDF terms into field elements is required.
Some features, however, are indirectly supported:
For example, proving that a literal is of a certain datatype is possible by only revealing the datatype of the literal while hiding the actual value.
This shows that the issue of encoding RDF terms to be used in ZKPs is not just an implementation detail but rather a fundamental consideration when designing a proof system.

Conditional expressions (\texttt{if}, \texttt{bound}, \texttt{coalesce}, ...) are currently considered out-of-scope to be proven in zero-knowledge. 
If corresponding variables are projected, then the corresponding properties can be shown to be correct.
Again, proving non-existence of information is not possible.

\begin{comment}
One exception in special cases is \texttt{IN}.
In a FILTER expression, {IN} may be proven using a proof of set membership and NOT IN may be proven using a proof of set non-membership\todo{JW: My comments related to MINUS + OWA also apply here}.
Such proofs may require a particular setup by an issuer of a signed dataset, which is the case for a Universal Accumulator~\cite{DBLP:journals/iacr/VittoB20}.
\end{comment}

Aggregates specify computations typically used for analytical queries. 
In the scope of this work, we consider proving \texttt{COUNT}, \texttt{SUM}, \texttt{AVG}, \texttt{MIN}, or \texttt{MAX} out-of-scope.
However, in general, we do see a potential approach to support aggregates using arithmetic circuits, a general-purpose cryptographic construct to specify and prove custom relations. 
% In designing these circuits compatible with the current approach of zkRDF, the number of input values to the aggregates would need to be disclosed.
% To prove such analytical queries, a computation-centric approach might be a better fit than the data-centric zkRDF approach.
We therefore consider this as future research.

If variables in aggregates are projected, then the underlying data is revealed and a verifier is able to compute the aggregates themselves, which then enables using aggregates in \texttt{HAVING} clauses to filter results of grouped solutions.

\paragraph{Property Paths.}
In scope of this work, we consider property paths to be out-of-scope.
However, in general, property paths are supported in zkRDF.
There does not exist a generic way to obtain the intermediate nodes of a path; so we outline a naive approach:
Beginning with the query's solution mappings, \ie the end of the path, iteratively retrace the query-defined path in the queried dataset until the start of the path is reached.
While we did not implement this in our proof-of-concept, proving the existence of a property path in a data-centric manner is possible.
The selectively disclosing dataset would reveal the property path while hiding intermediary nodes. 
Negated property sets are not supported as non-existence cannot be proven.

\paragraph{Solution Modifiers.} 
In zkRDF's data-centric approach, the application of solution modifiers is not proven, rather only the resulting solution mappings are disclosed.
For example, \texttt{GROUP BY} % Partitions the solution set into groups, which is a prerequisite for using aggregate functions.
is supported even when the variable to group by remains hidden. 
Matching RDF terms will be substituted with the corresponding blank nodes across a term's occurrences, which then allows to still group the solution mappings based on that variable which is now simply mapped to the stand-in blank node.

The result of \texttt{DISTINCT} (or \texttt{REDUCED}) is simply a selectively disclosing dataset that only proves the reduced set of solution mappings.
The fact that solution mappings are distinct is not proven; the data consumer verifies the property by inspection.

For \texttt{LIMIT}, the desired number of solution mappings is proven to exist; discarded solution mappings are simply omitted and not proven to exist. 
The desired cardinality of the solution is verified by inspection.

In the scope of this work, the result of \texttt{OFFSET} cannot be proven in zero-knowledge by zkRDF.
Without pre-scribing an ordering using \texttt{ORDER BY}, the solution is a set of solution mappings.
It is thus ambiguous which solution mapping would be considered by \texttt{OFFSET}.
If the set of solution mappings proven to exist is sufficiently large, then the data consumer can apply the offset themselves.
Such oversharing may reveal unnecessary information.

Similarly for \texttt{ORDER BY}: % Sorts the solutions based on one or more variables.
If the variable to order by is not projected, then the matching RDF terms will be replaced by blank nodes.
Proving an ordering in zero-knowledge is not possible by zkRDF in the scope of this work.
Similar to aggregates, there may be again some potential by using arithmetic circuits -- potential future research.
If the variable to order by is projected, then the matching RDF terms and the resulting ordering are revealed.
The data consumer verifies it by inspection.



\subsection{Performance: PoC Implementation}
\label{sec:eval_performance}

The related work mentioned on proving the execution of a SPARQL query in a zkVM (SPARQL-in-zkVM) provides a small testing benchmark\footnote{\url{https://github.com/jeswr/risc0-ed25519-zk-sparql\#running-the-benchmark}}, which we refer to as \texttt{risc0-bench}.
We emphasize that this testing benchmark is not a contribution of this paper.
%\todo{JW: IIRC you also benchmarked against \url{https://github.com/jeswr/zkSPARQL-bench} - I suggest including those results here.
%CB: but you did not... what is the point of including numbers with nothing to compare it to?}


The \texttt{risc0-bench} testing benchmark includes a dataset of four VCs with varying yet overlapping contents from different issuers.
The employed signature scheme is a \texttt{Ed25519} Linked Data Signature.
The VCs are provided in JSON-LD serialisation.
The \texttt{risc0-bench} testing benchmark offers three queries\footnote{\url{https://github.com/jeswr/risc0-ed25519-zk-sparql/blob/main/queries}}:
\begin{itemize}[noitemsep,topsep=0pt,parsep=0pt,partopsep=0pt,wide,labelwidth=!,labelindent=0pt,leftmargin=14pt]
    \item \texttt{can-drive}: a query with \texttt{FILTER} statements expressing numeric bounds.
    \item \texttt{employment-status}: a simple graph pattern query.
    \item\texttt{show-all}: a generic \enquote{select all} \texttt{spo} graph pattern query.
\end{itemize}
These queries do not include any aggregation functions but are instead rather geared towards querying credentials for selective disclosure of personal information.
This allows for a comparison of the computation-centric approach  with our data-centric zkRDF.

\paragraph{Performance Comparison.} We summarize our benchmarking results in Table~\ref{tab:framework_comparison}. 
We ran \texttt{risc0-bench} for the SPARQL in zkVM approach on our hardware, a consumer laptop with an AMD Ryzen 7 PRO 5850U, to generate a performance baseline:
\begin{comment}
SPARQL zkVM         prove           verify
can-drive           1571816.15 ms   7792.54 ms
employment-status   1463066.24 ms   6782.48 ms
show-all            1736362.18 ms   8878.46 ms
\end{comment}
\begin{comment}
\begin{table}[htbp]
    \centering
    \caption{Performance of the SPARQL-zkVM approach (in milliseconds)}
    \label{tab:sparql_zkvm_performance}
    \begin{tabular}{l|r|r}
        \toprule
        \textbf{Task} & \textbf{Prove Time} & \textbf{Verify Time} \\
        \midrule
        can-drive         & 1571816.15 & 7792.54 \\
        employment-status & 1463066.24 & 6782.48 \\
        show-all          & 1736362.18 & 8878.46 \\
        \bottomrule
    \end{tabular}
\end{table}
\end{comment}
Proving the execution of the respective queries using a zkVM thus takes approximately 26 minutes (\texttt{can-drive}), 24 minutes (\texttt{employment-status}), and 29 minutes (\texttt{show-all}).

\begin{table}[tb]
    \centering
    \caption{Performance comparison in milliseconds (ms).}
    \label{tab:framework_comparison}
    \small 
    \setlength{\tabcolsep}{3pt} 

    \begin{tabular}{|l|rr|rr|}
        \toprule
        & \multicolumn{2}{c|}{%
            \shortstack{SPARQL-in-zkVM (theirs)}
          } 
        & \multicolumn{2}{c|}{%
            \shortstack{zkRDF (ours)}
          } \\
        \textbf{Task} & \textbf{Prove} & \textbf{Verify} & \textbf{Prove} & \textbf{Verify} \\
        \midrule
        can-drive       & 1571816.15 & 7792.54~~ & \hspace{10pt}141.61 & \hspace{10pt} 27.63   \\
        employment-status & 1463066.24 & 6782.48~~ & 5.51   & 5.38    \\
        show-all        & 1736362.18 & 8878.46~~ & 437.02 & 1650.80 \\
        \bottomrule
    \end{tabular}
\end{table}


Our implementation, based on the work provided in~\cite{DBLP:conf/esws/BraunK25}, uses the \texttt{BBS+} signature scheme instead of \texttt{Ed25519}.
We therefore created \texttt{BBS+} signatures for the credential data provided by \texttt{risc0-bench}.
Moreover, we ran into a practical issue during implementation:
How do we handle multiple VCs in our knowledge base?
More specifically, do we merge the default graphs or do we mint new named graphs for each VC's default graph.
As mentioned in Section~\ref{sec:prelim}, we went with the latter approach which required amending the queries accordingly to work on named graphs.
The modified version of \texttt{risc0-bench}\footnote{Contained in: \url{https://anonymous.4open.science/r/rdf-zkp-sparql/tree/main/benches}} results in the following performance numbers on our hardware (using \texttt{criterion.rs}).
\begin{comment}
zkRDF                prove           verify
can-drive                141.61 ms    27.63 ms
employment-status          5.51 ms     5.38 ms
show-all                 437.02 ms  1650.80 ms 
\end{comment}
\begin{comment}
\begin{table}[htbp]
    \centering
    \caption{Performance of the zkRDF approach (in milliseconds)}
    \label{tab:zkrdf_performance}
    \begin{tabular}{l|r|r}
        \toprule
        \textbf{Task} & \textbf{Prove Time} & \textbf{Verify Time} \\
        \midrule
        can-drive         & 141.61 & 27.63   \\
        employment-status & 5.51   & 5.38    \\
        show-all          & 437.02 & 1650.80 \\
        \bottomrule
    \end{tabular}
\end{table}
\end{comment}
For this exemplary benchmark, the data-centric approach of zkRDF is at least three orders of magnitude faster in creating a proof than the computation-centric approach of executing SPARQL within a zkVM.

A note on the (relatively) long verification time of the \texttt{show-all} query: 
Verifying the cryptographic proof only takes around 350ms; the rest is taken up by loading and handling the received dataset -- probably not efficiently implemented. %\todo{I would have expected the only thing needed here is the proof of knowledge of signature for the dataset, since everything in the dataset is disclosed - why is this still significantly longer than the verification time for emplyoment-status (5ms)}
Still, 350ms may seem long compared to the 5ms verification time of the \texttt{employment-status} query.
For comparison: The \texttt{employment-status} query entails proving the existence of one binding using a corresponding proof of knowledge of signature of one VC.
The \texttt{show-all} query entails proving the existence of 85 proving existence of 85 bindings using corresponding proofs of knowledge of signature of their respective VCs (\ie 85 proofs).
Although only four VC \enquote{produce} the 85 bindings, it must not be revealed which bindings are produced by a particular graph -- it should only be disclosed that they exist in \textit{some} graph. 
Whether that is the same or a different graph as for another binding must not be revealed.

Note that this behaviour is the same for BGPs consisting of multiple triple patterns as well:
Specifying such a BGP will result in a binding and corresponding proof that there exist RDF terms matching the BGP within the same graph.
If there exist another binding, there will be a corresponding proof but it will not be revealed if the second binding was produced by the same graph or a different graph as the first binding.

\paragraph{Performance Breakdown per Procedures.}
To further break down performance of the respective internal procedures from the abstract methodology presented in Section~\ref{sec:framework_abstract}, we tracked execution times within our code in addition to the \texttt{criterion.rs} benchmarking, presented in Table~\ref{tab:query_performance}. The total proving time may thus vary slightly compared to Table~\ref{tab:framework_comparison}.
The \texttt{Understand*} task is comprised of the {Query Execution} and parts of  the {Solution and Query Analysis} procedures. 
The remaining parts of the {Analysis} procedures are captured by \texttt{Specify Proof}, which is the \enquote{glue} between the analysis procedures and the proof system.
Finally, \texttt{Create Proof} refers to the generation of the cryptographic proof and \texttt{Present Proof} refers to the serialisation of the selectively disclosing dataset which together correspond to the Proof Creation procedure 
\begin{comment}
(ms)        understand* query, specify proof, create proof, present proof, total
can-drive   1.59,  1.36, 137.74,  0.90, 141.80
employ-stat 0.57,  0.38,   4.22,  0.23,   5.45
show-all    9.96, 30.88, 375.78, 18.60, 435.30
\end{comment}
\begin{table}[tb]
    \centering
    \caption{Query Performance Metrics in milliseconds (ms).}
    \label{tab:query_performance}

    \small % Use a smaller font
    \setlength{\tabcolsep}{3pt} % Reduce space between columns

    \begin{tabular}{|l||r|r|r|r||r|}
        \toprule
        \textbf{Query} & \shortstack{Under-\\stand*} & \shortstack{Specify\\Proof} & \shortstack{Create\\Proof} & \shortstack{Present\\Proof} & \textbf{Total} \\
        \midrule
        can-drive   & 1.59  & 1.36  & 137.74 & 0.90  & 141.80 \\
        employ-stat & 0.57  & 0.38  & 4.22   & 0.23  & 5.45   \\
        show-all    & 9.96  & 30.88 & 375.78 & 18.60 & 435.30 \\
        \bottomrule
    \end{tabular}
\end{table}


We thus see that cryptographic proof creation takes most of the overall execution time.
Executing the query, deriving which proofs to create and presenting the result in the end offer less optimization potential compared to choosing an efficient proof system. 

\subsection{Privacy \& Security: Discussion}
\label{sec:eval_security}
\paragraph{Information Leakage.}
The zkRDF approach offers significant enhancement of data privacy by reducing the amount of information being revealed.
Albeit named zero-knowledge (zk) RDF, the approach itself is not completely zero-knowledge.
For example, the number of proofs of knowledge of signatures indicate how many graphs from the underlying private dataset were used to create a solution mapping.
Moreover by these proofs of knowledge of signature, the size of the respective RDF graphs is being revealed.
This means that certain information may be revealed, even though ZKPs, which are indeed zero-knowledge, are being applied.

Such information leaks are due to a discrepancy between the intention of a proof and the actual relation being proven by an employed ZKP:
For example, consider the following intention: \textit{\enquote{I want to prove that a triple attesting my correct birthdate was signed by the government.}}
One way to instantiate the intended proof is by employing a proof of knowledge of signature on the RDF graph which contains such triple.
However, for multi-message signatures like BBS+~\cite{DBLP:journals/iacr/CamenischDL16}, the corresponding relation is: \textit{\enquote{I know a valid signature on a list of (secret) values; I reveal some values from that list to you.
Also, I prove that I know the remaining hidden values.}} %\todo{JW: is there a simpler way of saying this? CB: like so?} 
It is required to prove knowledge of hidden values in this case because the signature algorithm already exposes the length of the list of values.
Minimizing the information exposed might require more complex and  potential slower proof systems like the SPARQL-in-zkVM approach. 

For system architects, it is therefore crucial to consider the discrepancy between the theoretical intention of a proof and the choice of its practical implementation.
For some use cases, the outlined leaking of graph size might pose an acceptable trade-off.
For other use cases, optimizing for data minimization might be paramount.

\paragraph{Malicious Query.}
Being able to prove properties about an RDF dataset in zero-knowledge does not imply immunity against information over-exposure.
Consider the \texttt{show-all} query from the performance evaluation.
All data from the underlying dataset is being exposed, and in addition it is proven to be signed by the corresponding issuers.
An attacker thus not only gains all the information but also receives assurance of that information for free.
To protect against such malicious queries, 
system architects may choose to restrict the set of acceptable queries.
This can be done by directly defining a set of permissible queries in a general query language or choosing a query language with restricted expressivity, a domain-specific language (DSL) like the Digital Credentials Query Language (DCQL)~\cite{openID4VP} (cf. Section~\ref{sec:prelim}).

Nonetheless, system architects must be aware of this threat when considering their system's trust model.
For some use cases, a human's check on the data to be presented may be sufficient.
For other use cases, possible information over-exposure must be ruled out; a non-trivial task and potential future research.
In any case, we strongly recommend formally verifying the system's security~\cite{DBLP:conf/www/BraunHKM24}.
% \todo{JW: I suggest doing a more formal security analysis of the zkRDF system which describes precisely what information is disclosed beyond the query results.}


% \todo{JW: Suggest making a statement about the following properties - ideally with formal proof. There may also be other properties discussed in the W3C VC security considerations section that you may wish to account for \url{https://www.w3.org/TR/vc-data-model-2.0/\#replay-attack} and \url{https://www.w3.org/TR/vc-di-bbs/\#privacy-considerations}}

\begin{comment}
These I would consider most important from a formal alaysis standpoint
\begin{enumerate}
    \item \textbf{Unlink-ability (pre and post quantum)}
    \item \textbf{Disclosure of sources/credentials used in a result}
    \item \textbf{post-quantum forgery: is it possible forge credentials and hence query results in a post quantum world}
    \item \textbf{post-quantum snooping: is it possible to reveal facts from the ZKP in a post-quantum world, is it possible to e.g. find the hash of the credential and work out the contents in a post quantum world}
\end{enumerate}

These I would consider useful discussion from a deployment / spec standpoint
\begin{enumerate}
    \item \textbf{Validity period}: Can/do you include validity periods in your result (derived from the validity period of the credentials from which the data was queried). Note this in turn could result in information leakage as to which credential the data was derived from
\end{enumerate}
\end{comment}

% \section{Discussion}

% \subsection{Data Model}

% \subsection{Standardization?}
\section{A Vision for Zero-Knowledge SPARQL}
\label{sec:conclusion}

We presented zkRDF, a data-centric approach to selectively disclose and prove SPARQL query results in RDF datasets.
% We believe zkRDF's abstract methodology to be general enough to also potentially apply to conceptual instantiations\todo{it  unclear to me what this means} other than the one presented in this paper.
zkRDF introduces the general idea and capability to prove properties about RDF data.

Using SPARQL queries to specify what properties need to be proven comes as a natural extension of RDF-based semantics for selective disclosure~\cite{DBLP:conf/esws/BraunK25}.
The current alternative from the VC domain, the Digital Crednetial Query Langauge, is deliberately designed to restrict what query can be expressed.
On one hand, this protects users from overly eager queries leading to data oversharing; but on the other hand, the low expressivity deniess expressing more complex properties like numeric bound.
In the context of the eIDAS regulation~\cite{eIDAS} for example, qualified trust service providers (qTSP) of electronic attestation of attributes (EAA) might find SPARQL's and thus zkRDF's expressivity useful to provide better services.
And, use cases along the lines of our initial example from Figure~\ref{fig:example}, this expressivity is strictly required to facilitate the use case.

For other more analytical use cases, the data-centric approach of zkRDF may not suffice.
As discussed in Section~\ref{sec:eval_competence}, support of aggregates comes with potential information leakage as pointed out in Section~\ref{sec:eval_security}.
When \eg such information exposure is unacceptable, a computation-centric approach may be the better choice.
Computation-centric approaches to querying RDF datasets in zero-knowledge thus pose as clear future research.
We thus declare our vision: proving SPARQL query results in \textit{true} zero-knowledge.

With our work on zkRDF, 
we add to the emergent research at the intersection of semantic web technologies and privacy-enhancing technology of ZKPs.
Given that sharing data via the Web increasingly requires attestation and assurance, we envision that the trinity of RDF as a data model, SPARQL as the query language and ZKPs as a means of privacy-preserving provenance  will be a valuable part of the solution to foster transparency and privacy, accountability, and -- ultimately -- trust on the Web.







%%
%% The acknowledgments section is defined using the "acks" environment
%% (and NOT an unnumbered section). This ensures the proper
%% identification of the section in the article metadata, and the
%% consistent spelling of the heading.
% \begin{acks}
% This research was partially funded by the Deutsche Forschungsgemeinschaft (DFG, German Research Foundation) – Project number 459291153 – Research Unit 5339.
% \end{acks}
%
% ---- Bibliography ----
%
% BibTeX users should specify bibliography style 'splncs04'.
% References will then be sorted and formatted in the correct style.
%
\bibliographystyle{splncs04}
\bibliography{literature}
%

\appendix

 \newpage
\section{Common Formalisation of RDF Graphs and Datasets}
\label{appendix:rdf}

We re-use the common formalisation as in~\cite{DBLP:conf/esws/BraunK25}:
Let $\mathcal{U}$ denote the set of all HTTP URIs~\cite{uri,http}, $\mathcal{B}$ the set of all blank nodes, and $\mathcal{L}$ the set of all literals.
Let $\mathcal{G}$ denote the set of all RDF graphs.
An RDF graph $G \in \mathcal{G}$ is defined as a set of triples.
A triple $t$ is defined as $t \in ( \mathcal{U} \cup \mathcal{B}) \times \mathcal{U} \times (\mathcal{U} \cup \mathcal{B} \cup \mathcal{L})$.
%
Let $\mathcal{D}$ denote the set of all RDF datasets.
An RDF dataset ${D} \in \mathcal{D}$ is a set of named graphs and an unnamed \textit{default graph}.
A named graph~\cite{DBLP:conf/www/CarrollBHS05} is a pair $(n, G_n)$ where $n \in (\mathcal{U} \cup \mathcal{B})$ and $G_n \in \mathcal{G}$.
% \todo{JW: Note that this diverges from the definition in \url{https://www.w3.org/TR/rdf11-datasets/#the-graph-name-denotes-the-named-graph-or-the-graph} which mandates named graphs have IRI names. CB: Your reference refers to their interpretation - different story. See the definition in the syntax: https://www.w3.org/TR/rdf11-concepts/\#section-dataset }
The default graph does not have a graph name, $(\_, G)$ with $G \in \mathcal{G}$.
%
A triple belonging to a graph $G_n$ within an RDF dataset is also referred to as a quad $q$ with $q \in ( \mathcal{U} \cup \mathcal{B}) \times \mathcal{U} \times (\mathcal{U} \cup \mathcal{B} \cup \mathcal{L}) \times \{n\}$, where $n$ is the graph name.

\section{SPARQL Formalisation Foundations}
\label{appendix:sparql}

Let $\mathcal{V}$ denote the set of all variables, disjoint from $\mathcal{U}$, $\mathcal{B}$, and $\mathcal{L}$.
A Basic Graph Pattern (BGP) is a set of triple patterns.
A {triple pattern} $\textit{tp}$ is defined as $\textit{tp} \in ( \mathcal{U} \cup \mathcal{B} \cup \mathcal{V}) \times (\mathcal{U} \cup \mathcal{V}) \times (\mathcal{U} \cup \mathcal{B} \cup \mathcal{L} \cup \mathcal{V})$.

The evaluation of a graph pattern against an RDF graph yields a multiset $\Omega$ of solution mappings.
A solution mapping $\mu \in \Omega$ is a partial function $\mu: \mathcal{V} \to (\mathcal{U} \cup \mathcal{B} \cup \mathcal{L})$.
The semantics of this evaluation are formally defined by the SPARQL algebra~\cite{SPARQL}.

A key operator for datasets is the \texttt{GRAPH} clause, which restricts the evaluation of a pattern $P$ to a specific graph. If the graph identifier $n$ is a URI, $P$ is evaluated against that named graph; if $n$ is a variable, $P$ is evaluated against every named graph in the dataset, and each resulting solution mapping is extended with a binding of $n$ to the name of the graph that produced the solution.

An expression $E$ in SPARQL is a formula that can be evaluated on a solution mapping $\mu$ to yield an RDF term or an error. 
$\mathcal{E}$ includes \eg relational expressions which can be used to express numeric bounds.
The algebra then defines operators on solution multisets, such as \texttt{FILTER}. 
The \texttt{FILTER} operator evaluates an expression $E \in \mathcal{E}$ for each solution mapping $\mu \in \Omega$ and retains a mapping if and only if the effective boolean value is \texttt{true}. 
Formally, if $\texttt{eval}(E, \mu)$ is the evaluation of $E$ for a mapping $\mu$, then:
$$ \texttt{FILTER}(\Omega, E) = \{ \mu \mid \mu \in \Omega \land \texttt{eval}(E, \mu) \}$$ % \todo{JW: remove "= true"}

The projection of $\Omega$ onto a set of variables $V_{\textit{proj} \subseteq \mathcal{V}}$ is a second multiset $\Omega'$ of mappings $\mu' \in \Omega'$ defined as:
\begin{multline*}
\Omega' = \texttt{PROJECT}(V_{\textit{proj}}, \Omega) = \{ \mu' \mid   \exists \, \mu \in \Omega \text{ s.t. } \\ \hspace{15pt} (\text{dom}(\mu') = \text{dom}(\mu) \cap V_{\textit{proj}}) \land (\forall ?v \in \text{dom}(\mu'), \mu'(?v) = \mu(?v)) \} 
\end{multline*}



\section{A Formalisation of zkRDF}
\label{appendix:zkrdf}

While the main body of this paper explains our approach in prose for broader accessibility (Section~\ref{sec:framework}), this section provides an initial description of the formal underpinnings of zkRDF. 

\paragraph{Canonicalised RDF Graphs and Datasets.}
If not canonicalised, digital signatures on RDF graphs are dependant on the particular bitstring signed, \ie serialisation and the order of triples matter.
RDF graphs need to be canonicalised~\cite{RDFcanon} prior to signature creation or verification such that the resulting Linked Data Signatures cover the RDF graph itself, independent of its serialisation.
%\todo{JW: It is how we do it, not required, you can always just tell the verifier an ordering. CB: How do you do that?}
For our purpose, canonicalisation provides an ordering to a graph's triples.
More formally, we define a canonicalised RDF graph $G^+ \in \mathcal{G}$ as a list of triples such that each triple is assigned an index $i$, \ie $t_i = G^+[i]$.
As a triple is also an ordered sequence of RDF terms, we can then assign an index tuple to specific occurrences of RDF terms within their triple. % \todo{Since tou already reference the RDFC14n spec (\url{https://www.w3.org/TR/rdf-canon/}) I would cut back the re-definition here to say "canonicalisation provides a deterministic ordering of triples in a dataset, and identifiers for blank nodes".}
For example, the subject $s$ of a triple $t_0$ is $s = t_0[0] = G^+[0][0]$ and the subject's index tuple is thus denoted by $(0,0)$, \ie $(\textit{triple\_index},\textit{position\_index})$.
The same idea applies to RDF datasets with their quads as well.

We note that there exists a hybrid version of an RDF dataset where all its graphs are canonicalised \enquote{on their own} and the dataset only contains these canonicalised versions without being canonicalised itself.
As pointed out in Sections~\ref{sec:prelim} and~\ref{sec:eval_performance}, we choose to operate on named graphs, \ie, minting a new graph name for each VC's default graph, instead of merging the VC's default graph into the query engines default graph.
We found this approach to be convenient for tracking which term occurred in which particular Verifiable Credential, \ie a canonicalised RDF graph, resulting in an index triple of $(\texttt{graph\_name},\texttt{triple\_index},\texttt{position\_index})$ per occurrence of a term.
More formally, let $I \in ((\mathcal{U} \cup \mathcal{B} )\times \mathbb{N} \times \mathbb{N})$ denote an \textit{index tuple}.
% The first element is the graph name $n \in (\mathcal{U} \cup \mathcal{B} )$\todo{the second is the index of the triple in the canonicalised dataset, and the position index of the term in the triple}.

\paragraph{Extending SPARQL Query Evaluation with Occurrence Tracing.}
In order to selectively disclose RDF terms from an RDF graph, we need to be able to identify the particular occurrences of that term which should be revealed.
In Section~\ref{sec:framework_instantiation}, we described the particluar approach of query-retracing, \ie, finding occurrences of an RDF term based on the query and its solution.

More formally, we extend the evaluation of a SPARQL query to directly obtain the index tuple of a variable-mapped RDF term.
Let $\texttt{eval}^+$
be the indexed evaluation function which evaluates a SPARQL algebraic expression against a canonicalised, \ie indexable, RDF graph.

The result of this evaluation is a multiset $\Omega^+$ of {indexed solution mappings}.
An \textit{indexed solution mapping} $\mu^+$ is a partial function that extends the standard mapping to include a set of index tuples for each variable binding, representing its provenance:
$$ \mu^+: \mathcal{V} \to (\mathcal{U} \cup \mathcal{B} \cup \mathcal{L}) \times \mathcal{P}((\mathcal{U} \cup \mathcal{B} )\times \mathbb{N} \times \mathbb{N}) $$
Each binding in $\mu^+$ for a variable $?v$ is a pair $(T, I)$, where $T \in (\mathcal{U} \cup \mathcal{B} \cup \mathcal{L})$ is an RDF term and $I \in \mathcal{P}((\mathcal{U} \cup \mathcal{B} )\times \mathbb{N} \times \mathbb{N})$ is a set of index tuples where the binding occurred.
For a solution mapping $\mu^+(?v) = (T, I)$, we use the dot-notation to access its members, \ie, $\mu^+(?v).T$ to refer to the mapped term and $\mu^+(?v).I$ to refer to the set of index tuples, \ie $\mu^+(?v) = (\,\mu^+(?v).T \,,\, \mu^+(?v).I\,)$.

\medskip

The semantics of $\texttt{eval}^+$ are realized by a corresponding set of indexed algebraic operators (\eg, $\texttt{JOIN}^+$, $\texttt{FILTER}^+$), which are defined to propagate and merge these sets of index tuples throughout the evaluation process.

Compatibility of two indexed solution mappings is not affected by the addition of the index tuple sets.
Two indexed solution mappings, $\mu_1^+$ and $\mu_2^+$, are compatible if, for every variable $?v$ in the intersection of their domains, they map it to the same RDF term. 
% \todo{I don't understand when they would have different domains - surely this is only applicable when we are talking about the same query; and thus the same domain of variables is in play?}
\begin{multline*}
    \texttt{compat}^+(\mu_1^+, \mu_2^+) \iff \\
    \forall ?v \in (\text{dom}(\mu_1^+) \cap \text{dom}(\mu_2^+)) \,,\, \mu_1^+(?v).T = \mu_2^+(?v).T
\end{multline*}
%\todo{I would use $:=$ as define-equals rather than $\iff$}

The $\text{merge}^+(\mu_1^+, \mu_2^+)$ function produces a new mapping $\mu'^+$ where for any variable $?v$:
$$\mu'^+(?v) =
\begin{cases}
    (\mu_1^+(?v).T \,,\, \mu_1^+(?v).I \cup \mu_2^+(?v).I) \\ \hspace{45pt} \text{if } ?v \in \text{dom}(\mu_1^+) \cap \text{dom}(\mu_2^+) \\
    \mu_1^+(?v) \hspace{20pt}  \text{if } ?v \in \text{dom}(\mu_1^+) \setminus \text{dom}(\mu_2^+) \\
    \mu_2^+(?v) \hspace{20pt}  \text{if } ?v \in \text{dom}(\mu_2^+) \setminus \text{dom}(\mu_1^+)
\end{cases}
$$

The indexed join of two solution sets, $\Omega_1^+$ and $\Omega_2^+$, is the multiset of all merged mappings from compatible pairs. The merge operation unions the index tuple sets for shared variables.
\begin{multline*}
    \texttt{JOIN}^+(\Omega_1^+, \Omega_2^+) = \\\hspace{20pt}
    \{ \text{merge}^+(\mu_1^+, \mu_2^+) \mid \mu_1^+ \in \Omega_1^+ \land \, \mu_2^+ \in \Omega_2^+ \land \, \texttt{compat}^+(\mu_1^+, \mu_2^+) \} 
\end{multline*}
All other graph pattern operations are defined in the same vein.

The rest of the SPARQL semantics remain the same, for example, the indexed filter is the same as the regular filter:
It applies an expression $E$ to a solution set $\Omega^+$. The expression is evaluated using only the term component ($T$) of each binding. Mappings for which the expression's effective boolean value is \texttt{true} are retained with their index sets ($I$) unmodified.
$$ \texttt{FILTER}^+(\Omega^+, E) = \{ \mu^+ \mid \mu^+ \in \Omega^+ \land \texttt{eval}(E, \mu^+) \} $$


\medskip
Extending SPARQL query evaluation in this way provides the necessary information to trace the origins of RDF terms that are mapped by any variable in-scope; from the variables to be projected, \ie $V_{\textit{proj}}$, and from those variables and blank nodes which are not to be projected.
The combination of occurrence information and projection information is required in order to determine, not only
which RDF terms to reveal or to hide.
More importantly -- it is used to determine which occurrences of hidden RDF terms need to be proven to be the same such that the graph pattern specified by the query is proven to be satisfied; even when specific nodes in the pattern remain hidden.



\paragraph{Deriving Relations, Statements, and Witnesses.}
SPARQL query components, \ie graph pattern $P$, expressions $E$ and projected variables $V_\textit{proj}$, and pre-projection solution set $\Omega^+$ provide all information to specify corresponding proofs about the underlying RDF dataset.
Let $r$ denote a relation that a proof system is able to prove.
Let $s$ denote a corresponding statement, the set of public inputs.
Let $w$ denote a corresponding witness, the set of secret inputs.
A proof specification is the tuple $(r,s,w) \in P_{\textit{spec}}$.

In the scope of this paper, we consider proofs of knowledge of signature (PoKS) and proofs of numeric bounds (PoNB), \ie $r \in \{\text{PoKS},\text{PoNB}\}$.
Of course, other proof systems that support proving other relations would require formalising their supported proofs as well.
We provide our formalisation as a sample to critique, improve and apply such formalisation to other relations as well.

\medskip

The {PoKS} relation proves knowledge of a valid signature for each graph that contributed to a solution. For each solution mapping, we identify all unique graph names present in its index tuples.
Formally, for each $\mu^+ \in \Omega^+$, let $N_{\mu^+} = \{ n \mid \exists \, (n, \_, \_) \in I \text{ for some } (T, I) \in \text{range}(\mu^+) \}$ denote the set of graphs that contain an occurrence of an RDF term mapped in $\mu^+$.

For each graph name $n \in N_{\mu^+}$, a corresponding proof specification $(\text{PoKS}, s, w)$ is derived.
Witness $w$ includes all variable-mapped occurrences of an RDF-term that belongs to the graph ($n$) and is not projected.
In addition, the signature $\sigma_w$ is part of the witness. 
Formally, the witness $w$ for a PoKS on graph $n$ is defined as:
$$ w = \{ \sigma_w,  I_w \} $$
where $I_w = \{ (j, k) \mid \exists \, ?v \notin V_{proj} \text{ s.t. } (n, j, k) \in \mu^+(?v).I \}$\todo{JW: There is inconsitency in the use of $?$ with variables. I would suggest consistently NOT using $?$ with variables.
CB: which ones did I miss? there is only ?v in use as a variable...}.
The statement includes all projected all variable-mapped occurrences of an RDF-term alongside other parameters $\rho_\sigma$ like the public verification key of the issuer.
Conversely, the public statement $s$ is defined as:
$$ s = \{ \rho_\sigma, I_s \} $$
where $I_s = \{ (j, k) \mid \exists \, ?v \in V_{proj} \text{ s.t. } (n, j, k) \in \mu^+(?v).I \}$.
The particular composition of witness and statement are dependent on the particular signature scheme used; we used BBS+~\cite{DBLP:journals/iacr/CamenischDL16}.

\medskip

The \text{PoNB} relation proves that a hidden numeric value satisfies a relational constraint specified in a \texttt{FILTER} clause. These proofs are derived for each solution mapping and for each applicable relational expression found in the query's set of expressions $E$.

Formally, let $E_{\textit{bounds}} \subseteq E$ be the subset of expressions of the form $(?v \text{ \texttt{op} } C)$, where $\texttt{op}$ is a relational operator (\eg, \texttt{>}, \texttt{<}) and $C$ is a constant numeric literal.
For each solution mapping $\mu^+ \in \Omega^+$ and for each expression $(?v \text{ op } C) \in E_{\textit{bounds}}$ where the variable $?v \notin V_{proj}$, a corresponding proof specification $(\text{PoNB}, s, w)$ is derived.

The public statement $s$ contains the public parameters of the numeric constraint:
% $$ s = \{ \texttt{op}, C \} $$
$$
s = \{(\text{min}, \text{max})\} =
\begin{cases}
  \{(C, +\infty)\} & \text{if } \texttt{op} \in \{ \texttt{>}\} \\
  \{(-\infty, C)\} & \text{if } \texttt{op} \in \{ \texttt{<} \}
\end{cases}
$$
Let the maximal and minimal value supported in a system be denoted by $+\infty$ and $-\infty$ respectively.
When there are multiple numeric bounds expressed for the same witness, relations may be aggregated resulting in fewer proofs to be created.

The witness $w$ contains the secret numeric value that satisfies the constraint:
$$ w = \{ \mu^+(?v).T \} $$

\medskip

To prove equality of hidden RDF terms, either within a graph pattern or between a node in a graph pattern and its numeric bounds, \ie between a PoKS and a PoNB, so-called equality constraints are specified.
In a particular solution mapping, each occurrence of a non-projected variable-mapped RDF term -- indicated by the set of index tuples -- needs to be proven equal.
This means that, within a solution mapping, two occurrences of the same hidden term are proven to be equal.
This does not mean that, across solution mappings, two occurrences of the same term are proven to be equal; this fact remains hidden.
Formally, let $C_{set}$ be the set of equality constraints derived from a single solution mapping $\mu^+$; defined as:
$$ C_{set} = \{ \mu^+(?v).I \mid ?v \in \text{dom}(\mu^+) \land ?v \notin V_{proj} \land |\mu^+(?v).I| > 1 \} $$

\medskip

\paragraph{Proof Creation and Presentation.}
The set of proof specifications $P_\textit{spec}$ and the set of equality constraints $C_\textit{set}$ are provided to the proof systems as input to create a corresponding proof $\pi$.
In our approach, this proof is a composite proof comprised of the single specified proofs.
Other proof systems may produce a single succinct proof.
To present the proof to a verifier, we construct a selectively disclosing dataset~\cite{DBLP:conf/esws/BraunK25}.
Other systems may choose a different way of presentation.

\begin{comment}

\section{Illustrating Example}
\label{appendix:example}

Example~\footnote{\url{https://github.com/uvdsl/rdf-zkp-sparql\#example}}

(assume prefixes to be set)

Assume there is a digitally signed dataset, e.g., a Verifiable Credential like
\lstset{style=turtle}
\begin{lstlisting}[
    language=Turtle, 
    caption={example A} ,
    label={listing:run-example-A}, 
    basicstyle=\scriptsize\ttfamily
]
# e.g. for an internship
[] a org:Membership;
  org:member <http://example.org/users#user123>;
  org:organization <http://example.org/organisations#aCompany> ;
  time:hasBeginning "2024-01-01T00:00:00Z"^^xsd:dateTimeStamp ;
  time:hasEnd "2025-12-31T23:59:59Z"^^xsd:dateTimeStamp .
# BBS+ signature by the issuer  and additional VC terms are omitted for brevity.
\end{lstlisting}
and another one like
\begin{lstlisting}[
    language=Turtle, 
    caption={example B} ,
    label={listing:run-example-B}, 
    basicstyle=\scriptsize\ttfamily
]
# to show range proof on integers work
<http://example.org/users#user123> foaf:age 25.
# BBS+ signature by the issuer and additional VC terms are omitted for brevity.
\end{lstlisting}
Now assume a SPARQL query (for the sake of the example) 
\begin{lstlisting}[
    language=Turtle, 
    caption={query Q} ,
    label={listing:run-example-Q}, 
    basicstyle=\scriptsize\ttfamily
]
SELECT ?org WHERE {
  ?user foaf:age ?age .
  ?membership org:member ?user.
  ?membership org:organization ?org .
  ?membership time:hasEnd ?endDate .
  FILTER(?endDate > xsd:dateTime("2025-10-01T00:00:00Z") )
  FILTER(?age <= 25 )
}
\end{lstlisting}
The full result can be found here\footnote{\url{https://github.com/uvdsl/rdf-zkp-sparql/blob/main/data/presentations/prover.trig}}, here an excerpt:
\begin{lstlisting}[
    language=Turtle, 
    caption={result P} ,
    label={listing:run-example-P}, 
    basicstyle=\scriptsize\ttfamily
]
GRAPH _:0_4 {
	_:0_0 _:0_1 _:0_2 .
	_:0_5 _:0_6 _:0_7 .
	_:0_10 time:hasEnd _:0_12 .
	_:0_10 org:member _:0_17 .
	_:0_10 org:organization <http://example.org/organisations#aCompany> .
}
GRAPH _:2_4 {
	_:0_17 foaf:age _:2_2 .
}
GRAPH _:presentationProofGraph {
	_:cproof rdf:type zkp:CompositeProof .
	_:cproof zkp:hasComponent _:p0 .
	_:cproof zkp:hasComponent _:p1 .
	_:cproof zkp:hasComponent _:p2 .
	_:cproof zkp:hasComponent _:p3 .
	_:p0 rdf:type bbsp16:PoKS16 .# proof of knowledge of signature
    _:p0 bbsp16:isProofOfKnowledgeOfSignatureOverGraph _:0_4 .
    # more proof details omitted for brevity
	_:p1 rdf:type lg16:LegoGroth16ProofOfRangeMembership . # numeric bounds proof
    _:p1 lg16:hasWitness _:0_12 .
	_:p1 lg16:hasLowerBound "1759276800"^^xsd:nonNegativeInteger .
	_:p1 lg16:hasUpperBound "18446744073709551615"^^xsd:nonNegativeInteger .
    # more proof details omitted for brevity
	_:p2 rdf:type bbsp16:PoKS16 .
	_:p3 rdf:type lg16:LegoGroth16ProofOfRangeMembership .
	# more proof details omitted for brevity
}
\end{lstlisting}
Note that \texttt{\_:0\_17} is re-used across graphs. The proof includes the fact that all occurrences of \texttt{\_:0\_17} refers to the same underlying secret value, the identifier of the user. To this end, two proofs of knowledge of signature are needed; one for each respective original graph of \texttt{\_:0\_4} and \texttt{\_:2\_4}. The two additional proofs are the proofs of numeric bounds of \texttt{\_:0\_12}, the end time of the membership, and of \texttt{\_:2\_2}, the age of the user.
Note that the company's identifier is revealed, as requested in the SELECT statement of the query.

All other information are only proven to be true:

    the signatures of the graphs are proven to be valid,
    the graphs' triples are proven to be known, without needing to reveale terms.
    the numeric bounds as per the FILTER statements are proven to be fulfilled without revealing the literal.
    the relations between user and organisation are proven to exist without revealing the membership identifier

Only information explicitly requested (or previously known as the IRIs) in the query are revealed, while still being proven to have been signed. For example, note that in the first two triples of graph \texttt{\_:0\_4}, we do not reveal that their subjects are actually the same value that also underlies \texttt{\_:0\_10}.

\end{comment}

\end{document}
